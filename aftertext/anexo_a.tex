% ----------------------------------------------------------
\chapter{Declaração padrão para empresa ou laboratório}
% ----------------------------------------------------------
	\begin{snugshade}

	\begin{center}
	{\textbf{DECLARAÇÃO DE CONCORDÂNCIA COM AS CONDIÇÕES PARA O DESENVOLVIMENTO DO TCC NA INSTITUIÇÃO}}
	\end{center}
		
	\end{snugshade}

	\vspace{10pt}

	Declaro estar ciente das premissas para a realização de Trabalhos de Conclusão de Curso (TCC) de Ciências da Computação e Sistema de In\-for\-ma\-ções da UFSC, particularmente da necessidade de que se o TCC envolver o desenvolvimento de um software ou produto específico (ex: um protocolo, um método computacional, etc.) o código fonte e/ou documentação completa correspondente deverá ser entregue integralmente, como parte integrante do relatório final do TCC. 

	Ciente dessa condição básica, declaro estar de acordo com a realização do TCC identificado pelos dados apresentados a seguir.

	\vspace{20pt}


	%\noindent\resizebox{\textwidth}{!}{
		\begin{tabular}{|l|X p{8cm}|}
			%\begin{small}
				\hline
			     \textbf{Instituição} &  ECL/INE/CTC \\ \hline
			     \textbf{Nome do Responsável} &  José Luís Almada G{\"u}ntzel \\ \hline
			     \textbf{Cargo/Função} &  Prof. INE/CTC \\ \hline
			     \textbf{Fone de Contato} &  (48) 37216466 / (48) 999711982 \\ \hline
			     \textbf{Acadêmico} & Gabriela Furtado da Silveira \\ \hline
			     \textbf{Título do trabalho} & Análise da Correlação Entre Conteúdo de Vídeo e Qualidade de Interpolação de Quadros de Vídeos
 \\ \hline
			     \textbf{Curso} & Ciências da Computação/INE/UFSC \\ \hline

			%\end{small}
		\end{tabular}
	%}

	\vspace{40pt}

	\begin{flushright}

		Florianópolis, \today.
		
	\end{flushright}

	\vspace{20pt}


	\begin{center}
	    \parbox{7cm}{
	    \centering
	      \rule{6cm}{1pt}\\
	       \small \textbf{Professor Responsável}\\
	       Prof. Dr. José Luís Almada G{\"u}ntzel     
	    }
		\hfill
	\end{center}  



%São documentos não elaborados pelo autor que servem como fundamentação (mapas, leis, estatutos). Deve ser precedido da palavra ANEXO, identificada por letras maiúsculas consecutivas, travessão e pelo respectivo título. Utilizam-se letras maiúsculas dobradas quando esgotadas as letras do alfabeto. 
