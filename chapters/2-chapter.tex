\chapter{Planejamento}

As seções seguintes apresentam aspectos relacionados ao desenvolvimento deste trabalho, incluindo método de pesquisa, cronograma, custos, recursos humanos necessários, comunicação e, finalmente, possíveis riscos.

\section{Método de Pesquisa}

Primeiramente serão estudados aspectos relacionados aos descritores de imagem e vídeo, como questões teóricas relevantes, categorização, principais exemplos na literatura, suas possíveis aplicações e disponibilidade de acesso.
A partir dessa pesquisa inicial, serão escolhidos os descritores considerados mais adequados para a proposta.
Tomada essa decisão, talvez seja necessário implementar um ou mais descritores, sobretudo se não forem encontradas implementações prévias de fácil acesso ou na linguagem desejada.

Posteriormente serão realizados testes aplicando os descritores sobre quadros retirados de algum conjunto de vídeos de teste \textit{datasets} presente na literatura.
Há quatro opções principais de conjuntos que podem ser utilizados para essa finalidade: Vimeo90K, \ac{UGC}, \ac{UVG} e  \textit{Deep Video Deblurring}. O Vimeo90K \cite{vimeo90k}, o qual apresenta triplas ou quíntuplas de quadros obtidas de vídeos com conteúdo variado. O \ac{UGC} \cite{Wang2019YouTubeUGC} e o \ac{UVG} \cite{mercat2020uvg}, por sua vez, possuem vídeos \textit{raw} (sem compressão), o que é favorável para evitar interferência de ruídos de codificação iniciais nos resultados. O \ac{UVG} contém sequências com \ac{fps} alto, o que é favorável ao caso de uso proposto, enquanto o UGC conta com conteúdos diversificados, o que pode ser positivo para análise das características, porém não tem informações sobre \ac{fps}.
Por último, o conjunto desenvolvido no trabalho sobre \textit{Deep Video Deblurring} também inclui vídeos com \ac{fps} alto, no entanto são comprimidos.

Tendo obtido os dados com os descritores, será executado o \ac{DIVC} sobre as sequências.
O codificador a ser usado para implementação do modelo é o \ac{VVenC}, o qual é uma versão rápida do \ac{VVC}, o padrão de codificação híbrido estado da arte.
Já a parte de \ac{VFI} será realizada com o \ac{RIFE} \cite{rife}, pois é um método que atinge qualidade boa em um tempo de execução relativamente baixo, além de possuir uma implementação aberta.

Por último, serão empregadas métricas para avaliação dos resultados obtidos.
Algumas opções de métricas para avaliação de qualidade são \ac{PSNR} e \ac{SSIM}, as quais são métricas clássicas e amplamente utilizadas.
Também é possível realizar análise a partir da \ac{VMAF}, a qual foi desenvolvida com o objetivo de considerar mais características relacionadas à qualidade percebida pelo \ac{SVH}.
A medição da taxa de bits (\textit{bit rate}) é dada em kbps (quantidade de informação - kilobits, processados ou transferidos em um segundo), e pode ser utilizada em métricas relacionadas à eficiência de codificação.
Uma delas é \ac{BD-Rate} (baseada em PSNR ou $\text{SSIM}_\text{dB}$) \cite{sullivan2001_bd}, cujos valores permitem medir a mudança percentual ($\%$) de \textit{bit rate} em uma qualidade similar.
% Outra alternativa é medir através de \ac{BD-PSNR} e \ac{BD-$\text{SSIM}_\text{dB}$}.
Outra alternativa é medir através de \ac{BD-PSNR} e \ac{BD-SSIMdB}.
% Outra alternativa é medir através de \ac{BD-PSNR} e BD-$\text{SSIM}_\text{dB}$.


Portanto, ao término deste trabalho de conclusão de curso, espera-se contar com uma técnica funcional que permita aumentar a eficiência de codificação do modelo \ac{DIVC} de acordo com pelo menos uma das métricas listadas acima. Para isso, o \ac{DIVC} será aplicado de maneira adaptativa com base na adequação do conteúdo da cena, sendo essa avaliada com o auxílio de descritores.


\section{Cronograma}
% ----------------------------------------------------------

%\afazer{consertar o problema que ocorreu nas tabelas ao colocar fundo colorido em algumas células}
% ver melhor com orientador e coorientador.\\
% \noindent{
% 	\begin{minipage*}{1\textwidth}
% 	\centering
\noindent\resizebox{\textwidth}{!}{

	\rowcolors{0}{shadecolor}{shadecolor}
	\begin{tabular}{|X p{5cm}|c|c|c|c|c|c|c|c|c|c|c|c|}
		%begin{small}
		\hline
		%{\rowcolor{shadecolor}}
		%\multirow{2}{*}{\cellcolor{shadecolor}} \\
		\multirow{-1}{*}{Etapas}                                                 &
		%\cline{2-12}
		\multicolumn{12}{|c|}{Meses}                                                                                                                                                                                                                                                                                                                                                         \\ \hhline{|~|*{12}{-|}}
		%{\rowcolor{shadecolor}}
		                                                                         & 1                      & 2                      & 3                      & 4                      & 5                      & 6                      & 7                      & 8                      & 9                      & 10                     & 11                     & 12                     \\ \hline
		\hiderowcolors Fundamentação teórica sobre descritores                   &
		\cellcolor{shadecolor}                                                   & \cellcolor{shadecolor} &                        &                        &                        &                        &                        &                        &                        &                        &                        &                                                 \\ \hline

		\hiderowcolors Aplicação de descritores selecionados ao \ac{DIVC}        &                        & \cellcolor{shadecolor} & \cellcolor{shadecolor} &                        &                        &                        &                        &                        &                        &                        &                        &                        \\ \hline

		\hiderowcolors Análise da correlação entre os descritores e a eficiência &                        & \cellcolor{shadecolor} & \cellcolor{shadecolor} & \cellcolor{shadecolor} &                        &                        &                        &                        &                        &                        &                        &                        \\ \hline

		\hiderowcolors Desenvolvimento da utilização adaptativa do \ac{DIVC}     &                        &                        &                        & \cellcolor{shadecolor} & \cellcolor{shadecolor} & \cellcolor{shadecolor} &                        &                        &                        &                        &                        &                        \\ \hline

		\hiderowcolors Testes e coleta de dados                                  &                        &                        &                        &                        & \cellcolor{shadecolor} & \cellcolor{shadecolor} & \cellcolor{shadecolor} & \cellcolor{shadecolor} & \cellcolor{shadecolor} &                        &                        &                        \\ \hline

		\hiderowcolors Relatório TCC I                                           &                        &                        &                        &                        & \cellcolor{shadecolor} & \cellcolor{shadecolor} &                        &                        &                        &                        &                        &
		\\ \hline

		\hiderowcolors Análise dos dados coletados                               &                        &                        &                        &                        &                        &                        &                        &                        & \cellcolor{shadecolor} & \cellcolor{shadecolor} &                        &                        \\ \hline

		\hiderowcolors Redação do rascunho de TCC                                &                        &                        &                        &                        &                        &                        &                        & \cellcolor{shadecolor} & \cellcolor{shadecolor} & \cellcolor{shadecolor} &                        &                        \\ \hline

		\hiderowcolors Entrega do rascunho de TCC                                &                        &                        &                        &                        &                        &                        &                        &                        &                        & \cellcolor{shadecolor} &                        &                        \\ \hline

		\hiderowcolors Relatório TCC II                                          &                        &                        &                        &                        &                        &                        &                        &                        &                        & \cellcolor{shadecolor} & \cellcolor{shadecolor} &
		\\ \hline

		\hiderowcolors Ajustes finais                                            &                        &                        &                        &                        &                        &                        &                        &                        &                        &                        & \cellcolor{shadecolor} &                        \\ \hline

		\hiderowcolors Defesa pública                                            &                        &                        &                        &                        &                        &                        &                        &                        &                        &                        &                        & \cellcolor{shadecolor} \\ \hline
	\end{tabular}
}
% \end{minipage*}
% }\vspace{-.1cm}

\section{Custos}
%\vspace{-.1cm}
% \noindent\resizebox{\textwidth}{!}{

Nenhum custo de licença está relacionado ao projeto.
\\

\begin{tabular}{|X p{3cm}|c|c|c|}
	%\begin{small}
	\hline
	%\rowcolor{shadecolor}
	{\cellcolor{shadecolor}} Item & {\cellcolor{shadecolor}} Quant. & {\cellcolor{shadecolor}} Valor Un. (R\$) & {\cellcolor{shadecolor}} Total (R\$) \\ \hline
	\hline
	\multicolumn{4}{|c|}{Material de Consumo}                                                                                                         \\ \hline
	Folhas impressas              & 100                             & $0,20$                                   & $20,00$                              \\ \hline
	Internet Wi-FI (valor mensal) & 12                              & $100$                                    & $1200,00$                            \\ \hline
	\hline
	\multicolumn{4}{|c|}{Outros recursos e serviços}                                                                                                  \\ \hline
	\multicolumn{4}{|c|}{Reserva de Contingência}                                                                                                     \\ \hline
	                              &                                 &                                          & $500,00$                             \\ \hline
	\hline
	\multicolumn{4}{|c|}{Total}                                                                                                                       \\ \hline
	                              &                                 &                                          & $1720,00$                            \\ \hline

	%\end{small}
\end{tabular}

\section{Recursos Humanos}

% \noindent\resizebox{\textwidth}{!}{
\begin{tabular}{|c|c|}
	%\begin{small}
	\arrayrulecolor{white}
	\hline
	\arrayrulecolor{black}
	\hline
	\rowcolor{shadecolor}
	Nome                         & Função                              \\ \hline
	Gabriela Furtado da Silveira & Autor                               \\ \hline
	André Beims Bräscher         & Orientador                          \\ \hline
	José Luís A. Güntzel         & Professor Responsável/Co-orientador \\ \hline
	Ismael Seidel                & Membro da Banca                     \\ \hline
	A definir                    & Membro da Banca                     \\ \hline
	Renato Cislaghi              & Coordenador de Projetos             \\ \hline

	%\end{small}
\end{tabular}
% }


\section{Comunicação}

\noindent\resizebox{\textwidth}{!}{
	\begin{tabular}{|X p{3cm}|X p{3cm}|X p{3cm}|X p{3cm}|X p{3cm}|}
		%\begin{small}
		%\arrayrulecolor{white}
		%\hline
		%\arrayrulecolor{black}
		\hhline{|*{5}{-|}}
		%\rowcolor{shadecolor}
		%\rowcolor{white}
		{\cellcolor{shadecolor}} O que precisa ser comunicado & {\cellcolor{shadecolor}} Por quem & {\cellcolor{shadecolor}} Para quem                      & {\cellcolor{shadecolor}} Melhor forma de Comunicação & {\cellcolor{shadecolor}} Quando e com que frequência \\ \hhline{|*{5}{-|}}
		%\rowcolor{white}
		Ante-Projeto                                          & Autor                             & Coordenador de Projetos                                 & Por meio do site de projetos                         & Final de Cada Semestre                               \\ \hhline{|*{5}{-|}}
		%\rowcolor{white}
		Reuniões com orientador                               & Autor                             & Orientador                                              & Videoconferência                                     & Semanalmente                                         \\ \hhline{|*{5}{-|}}
		%\rowcolor{white}
		Revisão do TCC                                        & Autor                             & Orientador, membros da banca                            & Papel impresso e/ou pdf                              & Próximo da conclusão                                 \\ \hhline{|*{5}{-|}}
		%\rowcolor{white}
		Dúvidas                                               & Autor                             & Orientador, membros da banca ou Coordenador de Projetos & E-mail e/ou videoconferência                         & Conforme necessidade                                 \\ \hhline{|*{5}{-|}}

		%\end{small}
	\end{tabular}
}

\section{Riscos}

\noindent\resizebox{\textwidth}{!}{
	\begin{tabular}{|X p{2cm}|c|c|c|X p{3cm}|X p{3cm}|}
		%\begin{small}
		%\hline
		\hhline{|*{6}{-|}}
		%\rowcolor{shadecolor}
		{\cellcolor{shadecolor}} Risco & {\cellcolor{shadecolor}}Probabilidade & {\cellcolor{shadecolor}}Impacto & {\cellcolor{shadecolor}}Prioridade & {\cellcolor{shadecolor}}Estratégia de Resposta & {\cellcolor{shadecolor}}Ações Preventivas                     \\ \hhline{|*{6}{-|}}
		Perda de Dados                 & baixa                                 & alto                            & alta                               & Utilizar ferramenta de versionamento           & Geração de backups                                            \\ \hhline{|*{6}{-|}}
		Problemas de Saúde             & baixa                                 & alto                            & alta                               & Realizar tratamento                            & Manter bons hábitos e ser prudente                            \\ \hhline{|*{6}{-|}}
		Alteração no Tema              & baixa                                 & alto                            & alta                               & Buscar novo tema ou modificar escopo atual     & Manter interação constante com orientador                     \\ \hhline{|*{6}{-|}}
		Alteração no Cronograma        & média                                 & alto                            & alta                               & Realizar as adaptações necessárias             & Ser responsável e organizado e manter-se dentro do cronograma \\ \hhline{|*{6}{-|}}

		%\end{small}
	\end{tabular}
}