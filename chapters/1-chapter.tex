\chapter{Introdução}

% Propor a ideia de um sistema de detecção de câncer de pele por meio de imagens que possa ser utilizado por agentes de saúde
% Propor que o sistema seja baseado no LLaVa e que seja treinado com imagens de câncer de pele

% Explicar que o treinamento completo seria difícil, mas seria possível fazer um fine tuning com PEFT ou LoRa
% Indicar que o resultado esperado é um modelo que possa classificar lesões de pele com alta acurácia

Lesões de pele são indicativos de diversas doenças, como infecções de baixo risco ou até mesmo câncer. Cerca de 30\% dos tumores malignos registrados no Brasil são
causados pelo câncer de pele \cite{skin_cancer_in_brazil}. Os tipos mais comuns desta doença são o carcinoma basocelular e o carcinoma epidermoide, que mesmo tendo uma
baixa letalidade, podem causar sequelas expressivas por conta do tratamento. O melanoma é menos frequente, mas possui uma letalidade muito maior, causando 75\% das mortes
por câncer de pele.

% Tendo em vista esse cenário, fica evidente a importância da codificação de vídeos, a qual busca formas de armazenar, transmitir e reproduzir esse tipo de dado de forma otimizada.
% Nesse contexto, a abordagem mais consolidada é a codificação híbrida, a qual vem sendo parte essencial do estado da arte da área nas últimas décadas.
% Os padrões de codificação híbridos valem-se de diferentes técnicas para examinar quadros e regiões dentro dos quadros a procura de informações redundantes para o sistema visual humano. Com isso, eles exploram a taxa de compressão atingida para minimizar tanto quanto possível o armazenamento, mantendo ao máximo a qualidade.
% Contudo, o aumento da demanda e da concorrência observado recentemente tem levado ao desenvolvimento de algoritmos cada vez mais complexos, impactando no custo computacional e no tempo de execução.
% Dessa maneira, a cada geração, a eficiência de codificação vem duplicando, ao custo de um aumento na complexidade de cerca de 10X.
% Por consequência, técnicas alternativas têm sido propostas para tentar melhorar a eficiência de codificação, e uma delas é o \ac{DIVC} \cite{ours}.
% Esse modelo de codificação consiste em combinar o uso de um dos padrões híbridos com a realização de \ac{VFI}.

% \ac{VFI} é uma técnica para gerar ao menos um quadro intermediário ($Q_t$), tomando como referência pelo menos um quadro anterior ($Q_0$) e um posterior ($Q_1$), como mostra a Figura.
% Nesse caso, \textit{t} representa a posição temporal do quadro interpolado em relação aos quadros de referência, com $0<t<1$.
% Os quadros podem ser produzidos com diferentes propósitos, e as implementações mais recentes são baseadas na utilização de \acp{NN}.
% Uma das abordagens mais empregadas pode ser classificada como \textit{flow-based}, e consiste em computar o \textit{optical flow}, o qual é usado para relizar \textit{warping} sobre os quadros de referência.
% Um exemplo desse tipo de arquitetura foi proposto por \textcite{niklaus2020softmax}.

\section{Objetivos}

O objetivo geral deste trabalho é melhorar a eficiência de codificação do modelo \ac{DIVC} através da remoção de quadros de vídeos de forma adaptativa.
Para essa finalidade, pretende-se utilizar diferentes descritores de imagem e/ou vídeo para extrair características das sequências as quais tenham potencial influência na qualidade dos resultados obtidos com a aplicação do \ac{DIVC}.

\subsection*{Objetivos Específicos}

\begin{itemize}
    \item Determinar quais descritores de imagem/vídeo apresentam melhor correlação com a eficiência de codificação na utilização do método \ac{DIVC};
    \item Melhorar a eficiência de codificação do modelo \ac{DIVC} através da remoção adaptativa de quadros, a partir da avaliação dos mesmos por meio dos descritores escolhidos.
\end{itemize}
