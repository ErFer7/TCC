\documentclass[
	12pt,				% tamanho da fonte
	oneside,			% para impressão no anverso. Oposto a twoside
	a4paper,			% tamanho do papel. 
	chapter=TITLE,		% títulos de capítulos convertidos em letras maiúsculas
	section=TITLE,		% títulos de seções convertidos em letras maiúsculas
	english,			% idioma adicional para hifenização
	brazil  			% o último idioma é o principal do documento
	]{abntex2}

\usepackage{setup/ufscthesisA4-alf}
\usepackage{csquotes}
\usepackage[backend = biber, style = abnt]{biblatex}
\usepackage{graphicx}
\usepackage{color}
\usepackage{listings}
\usepackage{multirow}
\usepackage{tabularx}
\usepackage[table]{xcolor}
\usepackage{colortbl}
\usepackage{framed}
\usepackage{amssymb}
\usepackage{afterpage}
\usepackage{hhline}
\usepackage{enumitem}
\usepackage{latexsym}
\usepackage{lipsum}
\usepackage{setspace}
\usepackage{tabularray}

\definecolor{shadecolor}{rgb}{0.8,0.8,0.8}

\setlength\bibitemsep{\baselineskip}
\DeclareFieldFormat{url}{Disponível~em:\addspace\url{#1}}
\NewBibliographyString{sineloco}
\NewBibliographyString{sinenomine}
\DefineBibliographyStrings{brazil}{%
	sineloco     = {\mkbibemph{S\adddot l\adddot}},
	sinenomine   = {\mkbibemph{s\adddot n\adddot}},
	andothers    = {\mkbibemph{et\addabbrvspace al\adddot}},
	in			 = {\mkbibemph{In:}}
}

\addbibresource{aftertext/references.bib}

\DeclareSourcemap {
	\maps[datatype=bibtex] {
		\map {
			\step[fieldset=abstract, null]
			\step[fieldset=pagetotal, null]
		}
		\map {
			\pertype{inproceedings}
			\step[fieldset=venue, null]
			\step[fieldset=eventdate, null]
			\step[fieldset=eventtitle, null]
			\step[fieldset=isbn, null]
			\step[fieldset=volume, null]
		}
	}
}

% Definições gerais
\autor{Eric Fernandes Evaristo}
\titulo{Uso de uma MLLM para classificar lesões de pele}
\orientador[Orientador]{Aldo von Wangenheim, Prof. Dr. rer.nat.}  % TODO: Verificar os títulos
\ano{2024}
\local{Florianópolis}
\instituicaosigla{UFSC}
\instituicao{Universidade Federal de Santa Catarina}
\tipotrabalho{Proposta de Trabalho de Conclusão de Curso}
\formacao{Bacharel em Ciências da Computação}
\nivel{Bacharel}
\programa{Curso de Graduação em Ciências da Computação}
\centro{Centro Tecnológico}

\preambulo {
	\imprimirtipotrabalho~do~\imprimirprograma~do~\imprimircentro~da~\imprimirinstituicao~para~a~obtenção~do~título~de~\imprimirformacao.
}

% Configurações de aparência do PDF final
\definecolor{blue}{RGB}{41,5,195}
\makeatletter
\hypersetup{
		pdftitle={\@title},
		pdfauthor={\@author},
    	pdfsubject={\imprimirpreambulo},
	    pdfcreator={LaTeX with abnTeX2},
		pdfkeywords={ufsc, latex, abntex2},
		colorlinks=true,
    	linkcolor=black,
    	citecolor=black,
    	filecolor=black,
		urlcolor=black,
		bookmarksdepth=4
}
\makeatother

% Declaração das siglas
\siglalista{MLLMs}{\textit{Multimodal Large Language Models}}
\siglalista{LLM}{\textit{Large Language Model}}
\siglalista{LLaVa}{\textit{Large Language and Vision Assistant}}
\siglalista{CLIP ViT-L/14}{\textit{Contrastive Language-Image Pre-training ViT-L/14}}
\siglalista{PEFT}{\textit{Parameter-Efficient Fine-Tuning}}
\siglalista{LoRa}{\textit{Low-Rank Adaptation}}
\siglalista{TCC}{\textit{Trabalho de Conclusão de Curso}}

% Compila a lista de abreviaturas e siglas e a lista de símbolos
\makenoidxglossaries

% Compila o indice
\makeindex

% Início do documento
\begin{document}

% Seleciona o idioma do documento (conforme pacotes do babel)
\selectlanguage{brazil}

% Retira espaço extra obsoleto entre as frases.
\frenchspacing

% Espaçamento 1.5 entre linhas
\OnehalfSpacing

% Elementos pré-textuais
% ---
% Capa
% ---
\imprimircapa
% ---

% ---
% Folha de rosto
% (o * indica que haverá a ficha bibliográfica)
% ---
\imprimirfolhaderosto*
% ---

% ---
% Inserir a ficha bibliografica
% ---
% http://ficha.bu.ufsc.br/
%\begin{fichacatalografica}
%	\includepdf{beforetext/Ficha_Catalografica.pdf}
%\end{fichacatalografica}
% ---

% ---
% Inserir folha de aprovação
% ---
%\begin{folhadeaprovacao}
	%\OnehalfSpacing
	%\centering
	%\imprimirautor\\%
	%\vspace*{10pt}		
	%\textbf{\imprimirtitulo}%
	%\ifnotempty{\imprimirsubtitulo}{:~\imprimirsubtitulo}\\%
	%		\vspace*{31.5pt}%3\baselineskip
	%\vspace*{\baselineskip}
	%\begin{minipage}{\textwidth}
	%% ~do~\imprimirprograma~do~\imprimircentro~da~\imprimirinstituicao~para~a~obtenção~do~título~de~\imprimirformacao.
	%Este~\imprimirtipotrabalho~foi julgado adequado para obtenção do Título de “\imprimirformacao” e aprovado em sua forma final pelo~\imprimirprograma. \\
	%	\vspace*{\baselineskip}
	%\imprimirlocal, \imprimirdata. \\
	%\vspace*{2\baselineskip}
	%\assinatura{\OnehalfSpacing\imprimircoordenador \\ %\imprimircoordenadorRotulo~do Curso}
	%\vspace*{2\baselineskip}
	%\textbf{Banca Examinadora:} \\
	%\vspace*{\baselineskip}
	%\assinatura{\OnehalfSpacing\imprimirorientador \\ %\imprimirorientadorRotulo}
%	%\end{minipage}%
%	\vspace*{\baselineskip}
%	\assinatura{Prof.(a) xxxx, Dr(a).\\
%	Avaliador(a) \\
%	Instituição xxxx}

%	\vspace*{\baselineskip}
%	\assinatura{Prof.(a) xxxx, Dr(a).\\
%	Avaliador(a) \\
%	Instituição xxxx}


%\end{folhadeaprovacao}

\begin{folhadeaprovacao}

	%\begin{snugshade}
	%\begin{center}
	%{\textbf{\small{FOLHA DE APROVAÇÃO DE PROPOSTA DE TCC}}}
	%\end{center}
	%\end{snugshade}
    %\vspace{-22pt}
    
    
    \begin{quadro}[htb]
	\centering
	\label{qua:folha_aprov_title}
	%\caption{\label{qua:folha_aprov}Formatação do texto.}	
	\begin{tabular}{ p{\textwidth}}
		\hline
		\cellcolor{shadecolor} \textbf{\small{FOLHA DE APROVAÇÃO DE PROPOSTA DE TCC}}\\ \hline
   
	\end{tabular}
	%\fonte{\textcite{NBR14724:2011}.}
\end{quadro}

\vspace{-22pt}

\begin{quadro}[htb]
	\centering
	\label{qua:folha_aprov}
	%\caption{\label{qua:folha_aprov}Formatação do texto.}	
	\begin{tabular}{|l|p{11cm}|}
		\hline
		\textbf{Acadêmico} & \imprimirautor\\ \hline
		\textbf{Título do trabalho}        & \imprimirtitulo\\ \hline
		\textbf{Curso}          & \imprimirprograma\\ \hline
		\textbf{Área de Concentração}        & Compressão de Vídeo  \\ \hline    
	\end{tabular}
	%\fonte{\textcite{NBR14724:2011}.}
\end{quadro}

\vspace{-15pt}

	\noindent \textbf{Instruções para preenchimento pelo \small{ORIENTADOR DO TRABALHO}:}
	\begin{itemize}[leftmargin=*,noitemsep,topsep=0pt]
		\item[-] \small Para cada critério avaliado, assinale um X na coluna SIM apenas se considerado aprovado. Caso contrário, indique as alterações necessárias na coluna Observação.
	\end{itemize}

\vspace{5pt}

	\noindent\resizebox{\textwidth}{!}{
		\centering
		\begin{tabular}{|X p{6.5cm}|X p{0.7cm}|X p{1.5cm}|X p{0.7cm}|X p{1.5cm}|X p{5.2cm}|}
			%\begin{small}\hhline{*{1}{{|~}*{4}{\arrayrulecolor{shadecolor}|-}|~}
				\hline
				%\multirow{2}{*}{\textbf{Critérios}} &  \textbf{Sim} &  \textbf{Parcial} &  \textbf{Não} &  \textbf{Não se aplica} & \multirow{2}{*}{\textbf{Observação}}\\ \hline{>{\arrayrulecolor{shadecolor}}}
			    \cellcolor{shadecolor} & \multicolumn{4}{c|}{\cellcolor{shadecolor} \textbf{Aprovado}} & \cellcolor{shadecolor} \\ \hhline{*{1}{>{\arrayrulecolor{shadecolor}}-}*{4}{>{\arrayrulecolor{black}}|-}>{\arrayrulecolor{shadecolor}}|->{\arrayrulecolor{black}}}
			    \multirow{-1}{*}{\cellcolor{shadecolor} {\small \textbf{Critérios}}} & \cellcolor{shadecolor} {\small \textbf{Sim}} &  \cellcolor{shadecolor} {\small \textbf{Parcial}} &  \cellcolor{shadecolor} {\small \textbf{Não}} &  \cellcolor{shadecolor} {\small \textbf{Não se aplica}} & \multirow{-1}{*}{\cellcolor{shadecolor} {\small \textbf{Observação}}} \\ \hline
				{\small 1.O trabalho é adequado para um TCC no CCO/SIN (relevância/abrangência)?} & \cellcolor{shadecolor} X & \cellcolor{shadecolor}  & \cellcolor{shadecolor}  & \cellcolor{shadecolor}  & \\ \hline
				{\small 2.O título do trabalho é adequado?} & \cellcolor{shadecolor} X & \cellcolor{shadecolor}  & \cellcolor{shadecolor}  & \cellcolor{shadecolor}  & \\ \hline
				{\small 3.O tema de pesquisa está claramente descrito?} & \cellcolor{shadecolor} X & \cellcolor{shadecolor}  & \cellcolor{shadecolor}  & \cellcolor{shadecolor}  & \\ \hline
				{\small 4.O problema/hipóteses de pesquisa do trabalho está claramente identificado?} & \cellcolor{shadecolor} X & \cellcolor{shadecolor}  & \cellcolor{shadecolor}  & \cellcolor{shadecolor}  & \\ \hline
				{\small 5.A relevância da pesquisa é justificada?} & \cellcolor{shadecolor} X & \cellcolor{shadecolor}  & \cellcolor{shadecolor}  & \cellcolor{shadecolor}  & \\ \hline
				{\small 6.Os objetivos descrevem completa e claramente o que se pretende alcançar neste trabalho?} & \cellcolor{shadecolor} X & \cellcolor{shadecolor}  & \cellcolor{shadecolor}  & \cellcolor{shadecolor}  & \\ \hline
				{\small 7.É definido o método a ser adotado no trabalho? O método condiz com os objetivos e é adequado para um TCC? } & \cellcolor{shadecolor} X & \cellcolor{shadecolor}  & \cellcolor{shadecolor}  & \cellcolor{shadecolor}  & \\ \hline
				{\small 8.Foi definido um cronograma coerente com o método definido (indicando todas as atividades) e com as datas das entregas (p.ex.Projeto I, II, Defesa)?} & \cellcolor{shadecolor} X & \cellcolor{shadecolor}  & \cellcolor{shadecolor}  & \cellcolor{shadecolor}  & \\ \hline
				{\small 9.Foram identificados custos relativos à execução deste trabalho (se houver)? Haverá financiamento para estes custos?} & \cellcolor{shadecolor} X & \cellcolor{shadecolor}  & \cellcolor{shadecolor}  & \cellcolor{shadecolor}  & \\ \hline
				{\small 10.Foram identificados todos os envolvidos neste trabalho?} & \cellcolor{shadecolor} X & \cellcolor{shadecolor}  & \cellcolor{shadecolor}  & \cellcolor{shadecolor}  & \\ \hline
				{\small 11.As formas de comunicação foram definidas (ex: horários para orientação)?} & \cellcolor{shadecolor} X & \cellcolor{shadecolor}  & \cellcolor{shadecolor}  & \cellcolor{shadecolor}  & \\ \hline
				{\small 12.Riscos potenciais que podem causar desvios do plano foram identificados?} & \cellcolor{shadecolor} X & \cellcolor{shadecolor}  & \cellcolor{shadecolor}  & \cellcolor{shadecolor}  & \\ \hline
				{\small 13.Caso o TCC envolva a produção de um software ou outro tipo de produto e seja desenvolvido também como uma atividade	realizada numa empresa ou laboratório, consta da proposta uma declaração (Anexo A) de ciência e concordância com a entrega do código fonte e/ou documentação produzidos? } & \cellcolor{shadecolor} X & \cellcolor{shadecolor}  & \cellcolor{shadecolor}  & \cellcolor{shadecolor}  & \\ \hline

			%\end{small}
		\end{tabular}
	}

\vspace{5pt}

	\noindent\resizebox{\textwidth}{!}{
		\begin{tabular}{| p{3.4cm}| p{2.35cm}| p{1.4cm}| p{5.4cm}|}
				\hline
			    {\tiny \textbf{Avaliação}} &  \multicolumn{1}{l}{\textbf{$\boxtimes$ \tiny Aprovado}}  & \multicolumn{2}{c|}{\textbf{$\Box$ \tiny Não Aprovado}}  \\ \hline \hline

			    {\tiny \textbf{Professor Responsável}} &  {\tiny Prof. Dr. José Luís A. Güntzel} & {\tiny 13/11/2023} & \\ \hline
                {\tiny \textbf{Orientador}} &  {\tiny Me. André Beims Bräscher} & {\tiny 13/11/2023} & \\ \hline

		\end{tabular}
	}


\end{folhadeaprovacao}

% ---

% ---
% Dedicatória
% ---
%\begin{dedicatoria}
%	\vspace*{\fill}
%	\noindent
%	\begin{adjustwidth*}{}{5.5cm}     
%		Este trabalho é dedicado aos meus colegas de classe e aos meus queridos pais.
%	\end{adjustwidth*}
%\end{dedicatoria}
% ---

% ---
% Agradecimentos
% ---
%\begin{agradecimentos}
%	Inserir os agradecimentos aos colaboradores à execução do trabalho. 
	
%	Xxxxxxxxxxxxxxxxxxxxxxxxxxxxxxxxxxxxxxxxxxxxxxxxxxxxxxxxxxxxxxxx. 
%\end{agradecimentos}
% ---

% ---
% Epígrafe
% ---
%\begin{epigrafe}
%	\vspace*{\fill}
%	\begin{flushright}
%		\textit{``Texto da Epígrafe.\\
%			Citação relativa ao tema do trabalho.\\
%			É opcional. A epígrafe pode também aparecer\\
%			na abertura de cada seção ou capítulo.\\
%			Deve ser elaborada de acordo com a NBR 10520.''\\
%			(Autor da epígrafe, ano)}
%	\end{flushright}
%\end{epigrafe}
% ---

% ---
% RESUMOS
% ---

% resumo em português
\setlength{\absparsep}{18pt} % ajusta o espaçamento dos parágrafos do resumo
\begin{resumo}
	\SingleSpacing
	
	%No resumo são ressaltados o objetivo da pesquisa, o método utilizado, as discussões e os resultados com destaque apenas para os pontos principais. O resumo deve ser significativo, composto de uma sequência de frases concisas, afirmativas, e não de uma enumeração de tópicos. Não deve conter citações. Deve usar o verbo na voz ativa e na terceira pessoa do singular. O texto do resumo deve ser digitado, em um único bloco, sem espaço de parágrafo. O espaçamento entre linhas é simples e o tamanho da fonte é 12. Abaixo do resumo, informar as palavras-chave (palavras ou expressões significativas retiradas do texto) ou, termos retirados de thesaurus da área. Deve conter de 150 a 500 palavras. O resumo é elaborado de acordo com a NBR 6028.

    Considerando o aumento significativo observado no consumo e compartilhamento de vídeos nos últimos anos, é notória a demanda por técnicas que permitam armazenamento, transmissão e reprodução desse tipo de dado de forma cada vez mais otimizada.
    Para essa finalidade, a abordagem mais consolidada é a codificação de vídeo híbrida, que tem sido parte essencial da área de codificação de vídeo nas últimas décadas.
    Contudo, estima-se que a complexidade de algoritmos tradicionais híbridos venha aumentando em 10X para um ganho de desempenho de 2X, a cada geração. 
    Nesse contexto, surge a necessidade de buscar modelos alternativos de codificação, como o \ac{DIVC}. 
    O \ac{DIVC} combina o modelo de codificação híbrido com \ac{VFI}, porém tem desempenho variável de acordo com o conteúdo do vídeo.
    Portanto, o objetivo deste trabalho é propor uma técnica para melhorar sua eficiência de codificação a partir da remoção de quadros de forma adaptativa. Para isso, os quadros serão analisados utilizando descritores que apresentem melhor correlação com a eficiência do modelo.
    
    \textbf{Palavras-chave:} Codificação de vídeo. VVC. Eficiência de codificação. Descritores de imagem. VFI. Redes Neurais.
\end{resumo}

% resumo em inglês
%\begin{resumo}[Abstract]
%	\SingleSpacing
%	\begin{otherlanguage*}{english}
%		Resumo traduzido para outros idiomas, neste caso, inglês. Segue o formato do resumo feito na língua vernácula. As palavras-chave traduzidas, versão em língua estrangeira, são colocadas abaixo do texto precedidas pela expressão “Keywords”, separadas por ponto.
		
%		\textbf{Keywords}: Keyword 1. Keyword 2. Keyword 3.
%	\end{otherlanguage*}
%\end{resumo}

%% resumo em francês 
%\begin{resumo}[Résumé]
% \begin{otherlanguage*}{french}
%    Il s'agit d'un résumé en français.
% 
%   \textbf{Mots-clés}: latex. abntex. publication de textes.
% \end{otherlanguage*}
%\end{resumo}
%
%% resumo em espanhol
%\begin{resumo}[Resumen]
% \begin{otherlanguage*}{spanish}
%   Este es el resumen en español.
%  
%   \textbf{Palabras clave}: latex. abntex. publicación de textos.
% \end{otherlanguage*}
%\end{resumo}
%% ---

{%hidelinks
	\hypersetup{hidelinks}
	% ---
	% inserir lista de ilustrações
	% ---
	\pdfbookmark[0]{\listfigurename}{lof}
	% \listoffigures*
	\cleardoublepage
	% ---
	
	% ---
	% inserir lista de quadros
	% ---
	%\pdfbookmark[0]{\listofquadrosname}{loq}
	%\listofquadros*
	%\cleardoublepage
	% ---
	
	% ---
	% inserir lista de tabelas
	% ---
	%\pdfbookmark[0]{\listtablename}{lot}
	%\listoftables*
	%\cleardoublepage
	% ---
	
	% ---
	% inserir lista de abreviaturas e siglas (devem ser declarados no preambulo)
	% ---
	\imprimirlistadesiglas
	% ---
	
	% ---
	% inserir lista de símbolos (devem ser declarados no preambulo)
	% ---
	%\imprimirlistadesimbolos
	% ---
	
	% ---
	% inserir o sumario
	% ---
	\pdfbookmark[0]{\contentsname}{toc}
	\tableofcontents*
	\cleardoublepage
	
}%hidelinks
% ---

% Elementos textuais
\textual

% 1 - Introdução
\chapter{Introdução}

Lesões de pele são indicativos de diversas doenças, como infecções de baixo risco ou até mesmo câncer. Cerca de 30\% dos tumores malignos registrados no Brasil são
causados pelo câncer de pele \cite{skin_cancer_in_brazil}. Os tipos mais comuns desta doença são o carcinoma basocelular e o carcinoma epidermoide, que mesmo tendo uma
baixa letalidade, podem causar sequelas expressivas por conta do tratamento. O melanoma é menos frequente, mas possui uma letalidade muito maior, causando 75\% das mortes
por câncer de pele \cite{skin_cancer_screening}.

A detecção precoce do câncer de pele é essencial para a garantia da efetividade do tratamento, pois a taxa de sobrevida dos pacientes tende a cair ao longo do avanço
da doença \cite{skin_cancer_survival}. Observa-se que indivíduos com um grau menor de escolaridade têm um prognóstico pior, pois apresentam tumores em estágios mais
avançados \cite{skin_cancer_socioeconomic}. Esta correlação pode indicar que a falta de conhecimento sobre a doença ou problemas socioeconômicos podem resultar no atraso
do diagnóstico e consequentemente na redução da taxa de sucesso do tratamento.

Considerando este cenário, o desenvolvimento de um sistema automatizado de fácil utilização de classificação de lesões de pele pode ser útil na identificação precoce de
melanomas e outros tipos de lesões de pele. \ac{MLLMs} conseguem obter uma precisão significativa na classificação de imagens de diagnósticos médicos, o que torna viável
o uso destes modelos para o desenvolvimento deste trabalho \cite{mllm_success_rate}.

O \ac{LLaVa} foi escolhido como o modelo base devido à limitação de recursos para o treinamento de um novo modelo e por ter o código disponível publicamente. Este
modelo combina o codificador visual \ac{CLIP ViT-L/14} e o \ac{LLM} \textit{Vicuna} \cite{llava}.

Com este modelo, é possível realizar um \textit{fine-tuning}, adaptando-o para a classificação de lesões de pele. Com o objetivo utilizar os recursos computacionais
disponíveis eficientemente, o \textit{fine-tuning} será feito com técnicas baseadas em \ac{PEFT}. Este método permite a redução de número de parâmetros usados no processo,
resultando em uma utilização menor de recursos \cite{peft}. Especificamente, planeja-se utilizar métodos como \ac{LoRa} ou outros similares para a adaptação do modelo.

Com as adaptações feitas no \ac{LLaVa}, espera-se obter um sistema confiável de classificação de lesões de pele.

\section{Objetivos}

O objetivo deste trabalho é desenvolver um sistema de classificação de lesões de pele utilizando um \ac{LLM} multimodal. Serão utilizados métodos de \textit{fine-tuning}
para adaptar o modelo base. Além disso, planeja-se analisar a eficiência entre diferentes técnicas de adaptação.

\subsection*{Objetivos Específicos}

\begin{itemize}
    \item Avaliar a eficiência de diferentes métodos de \textit{fine-tuning} e a precisão dos modelos resultantes;
    \item Obter um modelo adaptado baseado no \ac{LLaVa} que possua uma precisão satisfatória na classificação de lesões de pele;
    \item Comparar o modelo adaptado com outros \ac{MLLMs}.
\end{itemize}


% 2 - Capítulo 2
\chapter{Planejamento}

As seções seguintes apresentam aspectos relacionados ao desenvolvimento deste trabalho, incluindo método de pesquisa, cronograma, custos, recursos humanos necessários, comunicação e, finalmente, possíveis riscos.

\section{Método de Pesquisa}

Primeiramente serão estudados aspectos relacionados aos descritores de imagem e vídeo, como questões teóricas relevantes, categorização, principais exemplos na literatura, suas possíveis aplicações e disponibilidade de acesso.
A partir dessa pesquisa inicial, serão escolhidos os descritores considerados mais adequados para a proposta.
Tomada essa decisão, talvez seja necessário implementar um ou mais descritores, sobretudo se não forem encontradas implementações prévias de fácil acesso ou na linguagem desejada.

Posteriormente serão realizados testes aplicando os descritores sobre quadros retirados de algum conjunto de vídeos de teste \textit{datasets} presente na literatura. 
Há quatro opções principais de conjuntos que podem ser utilizados para essa finalidade: Vimeo90K, \ac{UGC}, \ac{UVG} e  \textit{Deep Video Deblurring}. O Vimeo90K \cite{vimeo90k}, o qual apresenta triplas ou quíntuplas de quadros obtidas de vídeos com conteúdo variado. O \ac{UGC} \cite{Wang2019YouTubeUGC} e o \ac{UVG} \cite{mercat2020uvg}, por sua vez, possuem vídeos \textit{raw} (sem compressão), o que é favorável para evitar interferência de ruídos de codificação iniciais nos resultados. O \ac{UVG} contém sequências com \ac{fps} alto, o que é favorável ao caso de uso proposto, enquanto o UGC conta com conteúdos diversificados, o que pode ser positivo para análise das características, porém não tem informações sobre \ac{fps}.
Por último, o conjunto desenvolvido no trabalho sobre \textit{Deep Video Deblurring} também inclui vídeos com \ac{fps} alto, no entanto são comprimidos.

Tendo obtido os dados com os descritores, será executado o \ac{DIVC} sobre as sequências. 
O codificador a ser usado para implementação do modelo é o \ac{VVenC}, o qual é uma versão rápida do \ac{VVC}, o padrão de codificação híbrido estado da arte.
Já a parte de \ac{VFI} será realizada com o \ac{RIFE} \cite{rife}, pois é um método que atinge qualidade boa em um tempo de execução relativamente baixo, além de possuir uma implementação aberta.

Por último, serão empregadas métricas para avaliação dos resultados obtidos. 
Algumas opções de métricas para avaliação de qualidade são \ac{PSNR} e \ac{SSIM}, as quais são métricas clássicas e amplamente utilizadas. 
Também é possível realizar análise a partir da \ac{VMAF}, a qual foi desenvolvida com o objetivo de considerar mais características relacionadas à qualidade percebida pelo \ac{SVH}.
A medição da taxa de bits (\textit{bit rate}) é dada em kbps (quantidade de informação - kilobits, processados ou transferidos em um segundo), e pode ser utilizada em métricas relacionadas à eficiência de codificação. 
Uma delas é \ac{BD-Rate} (baseada em PSNR ou $\text{SSIM}_\text{dB}$) \cite{sullivan2001_bd}, cujos valores permitem medir a mudança percentual ($\%$) de \textit{bit rate} em uma qualidade similar. 
% Outra alternativa é medir através de \ac{BD-PSNR} e \ac{BD-$\text{SSIM}_\text{dB}$}.
Outra alternativa é medir através de \ac{BD-PSNR} e \ac{BD-SSIMdB}.
% Outra alternativa é medir através de \ac{BD-PSNR} e BD-$\text{SSIM}_\text{dB}$.


Portanto, ao término deste trabalho de conclusão de curso, espera-se contar com uma técnica funcional que permita aumentar a eficiência de codificação do modelo \ac{DIVC} de acordo com pelo menos uma das métricas listadas acima. Para isso, o \ac{DIVC} será aplicado de maneira adaptativa com base na adequação do conteúdo da cena, sendo essa avaliada com o auxílio de descritores. 


\section{Cronograma}
% ----------------------------------------------------------

%\afazer{consertar o problema que ocorreu nas tabelas ao colocar fundo colorido em algumas células}
% ver melhor com orientador e coorientador.\\
% \noindent{
% 	\begin{minipage*}{1\textwidth}
% 	\centering
	\noindent\resizebox{\textwidth}{!}{

	\rowcolors{0}{shadecolor}{shadecolor}
	\begin{tabular}{|X p{5cm}|c|c|c|c|c|c|c|c|c|c|c|c|}
		%begin{small}
		\hline
		%{\rowcolor{shadecolor}}
		%\multirow{2}{*}{\cellcolor{shadecolor}} \\
		\multirow{-1}{*}{Etapas} &
		%\cline{2-12}
		\multicolumn{12}{|c|}{Meses} \\ \hhline{|~|*{12}{-|}}
		%{\rowcolor{shadecolor}}
     & 1 & 2 & 3 & 4 & 5 & 6 & 7 & 8 & 9 & 10 & 11 & 12\\ \hline
	\hiderowcolors Fundamentação teórica sobre descritores &
    \cellcolor{shadecolor}&\cellcolor{shadecolor} & & & & & & & & & & \\ \hline
    
	\hiderowcolors Aplicação de descritores selecionados ao \ac{DIVC} & & \cellcolor{shadecolor} & \cellcolor{shadecolor} & & & & & & & & &	\\ \hline
 
	\hiderowcolors Análise da correlação entre os descritores e a eficiência & & \cellcolor{shadecolor} & \cellcolor{shadecolor} & \cellcolor{shadecolor} & & & & & & & & \\ \hline
	
    \hiderowcolors Desenvolvimento da utilização adaptativa do \ac{DIVC} & & & & \cellcolor{shadecolor} & \cellcolor{shadecolor} & \cellcolor{shadecolor} & & & & & & \\ \hline
    
	\hiderowcolors Testes e coleta de dados & & & & & \cellcolor{shadecolor} & \cellcolor{shadecolor} & \cellcolor{shadecolor} & \cellcolor{shadecolor} &      \cellcolor{shadecolor} & & & \\ \hline

	\hiderowcolors Relatório TCC I & & & & & \cellcolor{shadecolor} & \cellcolor{shadecolor} & & & & & & 
    \\ \hline
		
    \hiderowcolors Análise dos dados coletados & & & & & & & & & \cellcolor{shadecolor} & \cellcolor{shadecolor} & & \\ \hline
 
	\hiderowcolors Redação do rascunho de TCC & & & & & & & & \cellcolor{shadecolor} & \cellcolor{shadecolor} & \cellcolor{shadecolor} & & \\ \hline
 
	\hiderowcolors Entrega do rascunho de TCC & & & & & & & & & & \cellcolor{shadecolor} & & \\ \hline
 
	\hiderowcolors Relatório TCC II & & & & & & & & & & \cellcolor{shadecolor} & \cellcolor{shadecolor} & 
    \\ \hline
	
    \hiderowcolors Ajustes finais & & & & & & & & & & & \cellcolor{shadecolor} & \\ \hline
    
	\hiderowcolors Defesa pública & & & & & & & & & & & & \cellcolor{shadecolor} \\ \hline
 
	\end{tabular}
}
% \end{minipage*}
% }\vspace{-.1cm}

\section{Custos}
%\vspace{-.1cm}
% \noindent\resizebox{\textwidth}{!}{

Nenhum custo de licença está relacionado ao projeto.
\\

	\begin{tabular}{|X p{3cm}|c|c|c|}
		%\begin{small}
			\hline
			%\rowcolor{shadecolor}
		    {\cellcolor{shadecolor}} Item & {\cellcolor{shadecolor}} Quant. & {\cellcolor{shadecolor}} Valor Un. (R\$) & {\cellcolor{shadecolor}} Total (R\$) \\ \hline
		    \hline
		    \multicolumn{4}{|c|}{Material de Consumo} \\ \hline
		    Folhas impressas & 100 & $0,20$ & $20,00$ \\ \hline
            Internet Wi-FI (valor mensal) & 12 & $100$ & $1200,00$ \\ \hline
		    \hline
		    \multicolumn{4}{|c|}{Outros recursos e serviços} \\ \hline
		    \multicolumn{4}{|c|}{Reserva de Contingência} \\ \hline
		     &  &  & $500,00$ \\ \hline
		    \hline
		    \multicolumn{4}{|c|}{Total} \\ \hline
		     &  &  & $1720,00$ \\ \hline

		%\end{small}
	\end{tabular}
	
	\section{Recursos Humanos}

% \noindent\resizebox{\textwidth}{!}{
	\begin{tabular}{|c|c|}
		%\begin{small}
			\arrayrulecolor{white}
			\hline
			\arrayrulecolor{black}
			\hline
			\rowcolor{shadecolor}
		    Nome & Função \\ \hline
		    Gabriela Furtado da Silveira & Autor \\ \hline
                André Beims Bräscher & Orientador \\ \hline
		    José Luís A. Güntzel & Professor Responsável/Co-orientador \\ \hline
		    Ismael Seidel & Membro da Banca \\ \hline
		    A definir & Membro da Banca \\ \hline
		    Renato Cislaghi & Coordenador de Projetos \\ \hline

		%\end{small}
	\end{tabular}
% }


\section{Comunicação}

\noindent\resizebox{\textwidth}{!}{
	\begin{tabular}{|X p{3cm}|X p{3cm}|X p{3cm}|X p{3cm}|X p{3cm}|}
		%\begin{small}
			%\arrayrulecolor{white}
			%\hline
			%\arrayrulecolor{black}
			\hhline{|*{5}{-|}}
			%\rowcolor{shadecolor}
			%\rowcolor{white}
		    {\cellcolor{shadecolor}} O que precisa ser comunicado & {\cellcolor{shadecolor}} Por quem & {\cellcolor{shadecolor}} Para quem & {\cellcolor{shadecolor}} Melhor forma de Comunicação & {\cellcolor{shadecolor}} Quando e com que frequência \\ \hhline{|*{5}{-|}}
			%\rowcolor{white}
		    Ante-Projeto & Autor & Coordenador de Projetos & Por meio do site de projetos & Final de Cada Semestre \\ \hhline{|*{5}{-|}}
			%\rowcolor{white}
		    Reuniões com orientador & Autor & Orientador & Videoconferência & Semanalmente \\ \hhline{|*{5}{-|}}
			%\rowcolor{white}
		    Revisão do TCC & Autor & Orientador, membros da banca & Papel impresso e/ou pdf & Próximo da conclusão \\ \hhline{|*{5}{-|}}
			%\rowcolor{white}
		    Dúvidas & Autor & Orientador, membros da banca ou Coordenador de Projetos & E-mail e/ou videoconferência & Conforme necessidade \\ \hhline{|*{5}{-|}}

		%\end{small}
	\end{tabular}
}

\section{Riscos}

\noindent\resizebox{\textwidth}{!}{
	\begin{tabular}{|X p{2cm}|c|c|c|X p{3cm}|X p{3cm}|}
		%\begin{small}
			%\hline
			\hhline{|*{6}{-|}}
			%\rowcolor{shadecolor}
		    {\cellcolor{shadecolor}} Risco & {\cellcolor{shadecolor}}Probabilidade & {\cellcolor{shadecolor}}Impacto & {\cellcolor{shadecolor}}Prioridade & {\cellcolor{shadecolor}}Estratégia de Resposta & {\cellcolor{shadecolor}}Ações Preventivas \\ \hhline{|*{6}{-|}}
		    Perda de Dados & baixa & alto & alta & Utilizar ferramenta de versionamento & Geração de backups \\ \hhline{|*{6}{-|}}
		    Problemas de Saúde & baixa & alto & alta & Realizar tratamento & Manter bons hábitos e ser prudente \\ \hhline{|*{6}{-|}}
		    Alteração no Tema & baixa & alto & alta & Buscar novo tema ou modificar escopo atual & Manter interação constante com orientador\\ \hhline{|*{6}{-|}}
		    Alteração no Cronograma & média & alto & alta & Realizar as adaptações necessárias & Ser responsável e organizado e manter-se dentro do cronograma \\ \hhline{|*{6}{-|}}

		%\end{small}
	\end{tabular}
}

% 3 - Capítulo 3
% \include{chapters/3-chapter}

% 4 - Conclusão
%\phantompart
% \include{chapters/4-chapter}

% Elementos pós-textuais
\postextual

% Referências bibliográficas
\begingroup
\printbibliography[title=REFERÊNCIAS]
\endgroup

% Apêndices
%\begin{apendicesenv}
%	\partapendices* 
%	\input{aftertext/apendice_a}
%\end{apendicesenv}

\end{document}
