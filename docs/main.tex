\documentclass[
	12pt,				% tamanho da fonte
	oneside,			% para impressão no anverso. Oposto a twoside
	a4paper,			% tamanho do papel. 
	chapter=TITLE,		% títulos de capítulos convertidos em letras maiúsculas
	section=TITLE,		% títulos de seções convertidos em letras maiúsculas
	english,			% idioma adicional para hifenização
	brazil  			% o último idioma é o principal do documento
	]{abntex2}

\usepackage{setup/ufscthesisA4-alf}
\usepackage{csquotes}
\usepackage[backend = biber, style = abnt]{biblatex}
\usepackage{graphicx}
\usepackage{color}
\usepackage{listings}
\usepackage{multirow}
\usepackage{tabularx}
\usepackage[table]{xcolor}
\usepackage{colortbl}
\usepackage{framed}
\usepackage{amssymb}
\usepackage{afterpage}
\usepackage{hhline}
\usepackage{enumitem}
\usepackage{latexsym}
\usepackage{lipsum}
\usepackage{setspace}
\usepackage{tabularray}
\usepackage{zref-abspage}
\usepackage{mdframed}

\newmdenv[
  leftmargin=0.5cm,
  rightmargin=0.5cm,
  skipabove=1em,
  skipbelow=1em,
  linecolor=gray!90,
  linewidth=1pt,
  leftline=false,
  rightline=false
]{dialogue}

\newcommand{\speak}[1]{\textbf{#1:}}

\definecolor{shadecolor}{rgb}{0.8,0.8,0.8}

\setlength\bibitemsep{\baselineskip}
\DeclareFieldFormat{url}{Disponível~em:\addspace\url{#1}}
\NewBibliographyString{sineloco}
\NewBibliographyString{sinenomine}
\DefineBibliographyStrings{brazil}{%
	sineloco     = {\mkbibemph{S\adddot l\adddot}},
	sinenomine   = {\mkbibemph{s\adddot n\adddot}},
	andothers    = {\mkbibemph{et\addabbrvspace al\adddot}},
	in			 = {\mkbibemph{In:}}
}

\addbibresource{aftertext/references.bib}

\DeclareSourcemap {
	\maps[datatype=bibtex] {
		\map {
			\step[fieldset=abstract, null]
			\step[fieldset=pagetotal, null]
		}
		\map {
			\pertype{inproceedings}
			\step[fieldset=venue, null]
			\step[fieldset=eventdate, null]
			\step[fieldset=eventtitle, null]
			\step[fieldset=isbn, null]
			\step[fieldset=volume, null]
		}
	}
}

% Definições gerais
\autor{Eric Fernandes Evaristo}
\titulo{Fine-tuning do MLLM LLaMA para a Classificação de Lesões de Pele}  % TODO: Adicionar "e Geração de Laudos"
\orientador[Orientador]{Aldo von Wangenheim, Prof. Dr. rer.nat.}
\coorientador[Coorientador]{Rodrigo de Paula e Silva Ribeiro, MSc}
\ano{2025}
\local{Florianópolis}
\instituicaosigla{UFSC}
\instituicao{Universidade Federal de Santa Catarina}
\tipotrabalho{Proposta de Trabalho de Conclusão de Curso}
\formacao{Bacharel em Ciências da Computação}
\nivel{Bacharel}
\programa{Curso de Graduação em Ciências da Computação}
\centro{Centro Tecnológico}

\preambulo {
	\imprimirtipotrabalho~do~\imprimirprograma~do~\imprimircentro~da~\imprimirinstituicao~para~a~obtenção~do~título~de~\imprimirformacao.
}

% Configurações de aparência do PDF final
\definecolor{blue}{RGB}{41,5,195}
\makeatletter
\hypersetup{
		pdftitle={\@title},
		pdfauthor={\@author},
    	pdfsubject={\imprimirpreambulo},
	    pdfcreator={LaTeX with abnTeX2},
		pdfkeywords={ufsc, latex, abntex2},
		colorlinks=true,
    	linkcolor=black,
    	citecolor=black,
    	filecolor=black,
		urlcolor=black,
		bookmarksdepth=4
}
\makeatother

% Declaração das siglas
% TODO: Conferir se todas as siglas estão certas no resto do documento

\siglalistaplural{MLLM}{\textit{Multimodal Large Language Model}}{\textit{Multimodal Large Language Models}}
\siglalistaplural{LLM}{\textit{Large Language Model}}{\textit{Large Language Models}}
\siglalista{LLaMA}{\textit{Large Language Model Meta AI}}
\siglalista{PEFT}{\textit{Parameter-Efficient Fine-Tuning}}
\siglalista{LoRA}{\textit{Low-Rank Adaptation}}
\siglalista{QLoRA}{\textit{Quantized Low Rank Adaptation}}
\siglalista{TCC}{Trabalho de Conclusão de Curso}
\siglalista{CPNM}{cânceres de pele não melanoma}
\siglalista{CBC}{carcinoma basocelular}
\siglalista{CEC}{carcinoma epidermoide}
\siglalistaplural{IA}{Inteligência Artificial}{Inteligências Artificiais}
\siglalistaplural{ViT}{\textit{Vision Transformer}}{\textit{Vision Transformers}}
\siglalista{GPT}{\textit{Generative Pre-trained Transformer}}
\siglalista{PaLM}{\textit{Pathways Language Model}}
\siglalista{ReLU}{\textit{Rectified Linear Unit}}
\siglalista{REM}{Reconhecimento de Entidades Mencionadas}
\siglalista{EQA}{\textit{Extractive Question Answering}}
\siglalista{GELU}{\textit{Gaussian Error Linear Units}}
\siglalista{MLP}{\textit{Multilayer perceptron}}
\siglalista{RLHF}{\textit{Reinforcement Learning from Human Feedback}}
\siglalista{RLAIF}{\textit{Reinforcement Learning from AI Feedback}}
\siglalista{SFT}{\textit{Supervised Fine-Tuning}}
\siglalista{RMSNorm}{\textit{Root Mean Square Layer Normalization}}
\siglalista{SwiGLU}{\textit{Swish-Gated Linear Unit}}
\siglalista{RoPE}{\textit{Rotary Positional Embeddings}}
\siglalista{GQA}{\textit{Grouped Query Attention}}
\siglalista{ACS}{Agentes Comunitários de Saúde}
\siglalista{APE}{\textit{Absolute Positional Embeddings}}
\siglalista{RPE}{\textit{Relative Positional Embeddings}}
\siglalista{NF4}{\textit{4-bit Normal Float}}
\siglalistaplural{CNN}{\textit{Convolutional Neural Network}}{\textit{Convolutional Neural Networks}}
\siglalista{ISIC}{\textit{International Skin Imaging Collaboration}}
\siglalista{LLaVA}{\textit{Large Language and Vision Assistant}}
\siglalista{CLIP}{\textit{Contrastive Language-Image Pre-training}}
\siglalista{STT/SC}{Sistema Integrado Catarinense de Telemedicina e Telessaúde}
\siglalista{HAM10000}{\textit{Human Against Machine with 10000 training images}}
\siglalista{rsLoRA}{\textit{rank-stabilized LoRA}}
\siglalista{BF16}{\textit{16-bit Brain Floating point}}
\siglalista{Adam}{\textit{Adaptive Moment Estimation}}
\siglalista{AdamW}{\textit{Adam with decoupled weight decay}}


% Compila a lista de abreviaturas e siglas e a lista de símbolos
\makenoidxglossaries

% Compila o indice
\makeindex

% Início do documento
\begin{document}

% Seleciona o idioma do documento (conforme pacotes do babel)
\selectlanguage{brazil}

% Retira espaço extra obsoleto entre as frases.
\frenchspacing

% Espaçamento 1.5 entre linhas
\OnehalfSpacing

% Elementos pré-textuais
% ---
% Capa
% ---
\imprimircapa
% ---

% ---
% Folha de rosto
% (o * indica que haverá a ficha bibliográfica)
% ---
\imprimirfolhaderosto*
% ---

% ---
% Inserir a ficha bibliografica
% ---
% http://ficha.bu.ufsc.br/
%\begin{fichacatalografica}
%	\includepdf{beforetext/Ficha_Catalografica.pdf}
%\end{fichacatalografica}
% ---

% ---
% Inserir folha de aprovação
% ---
%\begin{folhadeaprovacao}
	%\OnehalfSpacing
	%\centering
	%\imprimirautor\\%
	%\vspace*{10pt}		
	%\textbf{\imprimirtitulo}%
	%\ifnotempty{\imprimirsubtitulo}{:~\imprimirsubtitulo}\\%
	%		\vspace*{31.5pt}%3\baselineskip
	%\vspace*{\baselineskip}
	%\begin{minipage}{\textwidth}
	%% ~do~\imprimirprograma~do~\imprimircentro~da~\imprimirinstituicao~para~a~obtenção~do~título~de~\imprimirformacao.
	%Este~\imprimirtipotrabalho~foi julgado adequado para obtenção do Título de “\imprimirformacao” e aprovado em sua forma final pelo~\imprimirprograma. \\
	%	\vspace*{\baselineskip}
	%\imprimirlocal, \imprimirdata. \\
	%\vspace*{2\baselineskip}
	%\assinatura{\OnehalfSpacing\imprimircoordenador \\ %\imprimircoordenadorRotulo~do Curso}
	%\vspace*{2\baselineskip}
	%\textbf{Banca Examinadora:} \\
	%\vspace*{\baselineskip}
	%\assinatura{\OnehalfSpacing\imprimirorientador \\ %\imprimirorientadorRotulo}
%	%\end{minipage}%
%	\vspace*{\baselineskip}
%	\assinatura{Prof.(a) xxxx, Dr(a).\\
%	Avaliador(a) \\
%	Instituição xxxx}

%	\vspace*{\baselineskip}
%	\assinatura{Prof.(a) xxxx, Dr(a).\\
%	Avaliador(a) \\
%	Instituição xxxx}


%\end{folhadeaprovacao}

\begin{folhadeaprovacao}

	%\begin{snugshade}
	%\begin{center}
	%{\textbf{\small{FOLHA DE APROVAÇÃO DE PROPOSTA DE TCC}}}
	%\end{center}
	%\end{snugshade}
    %\vspace{-22pt}
    
    
    \begin{quadro}[htb]
	\centering
	\label{qua:folha_aprov_title}
	%\caption{\label{qua:folha_aprov}Formatação do texto.}	
	\begin{tabular}{ p{\textwidth}}
		\hline
		\cellcolor{shadecolor} \textbf{\small{FOLHA DE APROVAÇÃO DE PROPOSTA DE TCC}}\\ \hline
   
	\end{tabular}
	%\fonte{\textcite{NBR14724:2011}.}
\end{quadro}

\vspace{-22pt}

\begin{quadro}[htb]
	\centering
	\label{qua:folha_aprov}
	%\caption{\label{qua:folha_aprov}Formatação do texto.}	
	\begin{tabular}{|l|p{11cm}|}
		\hline
		\textbf{Acadêmico} & \imprimirautor\\ \hline
		\textbf{Título do trabalho}        & \imprimirtitulo\\ \hline
		\textbf{Curso}          & \imprimirprograma\\ \hline
		\textbf{Área de Concentração}        & Compressão de Vídeo  \\ \hline    
	\end{tabular}
	%\fonte{\textcite{NBR14724:2011}.}
\end{quadro}

\vspace{-15pt}

	\noindent \textbf{Instruções para preenchimento pelo \small{ORIENTADOR DO TRABALHO}:}
	\begin{itemize}[leftmargin=*,noitemsep,topsep=0pt]
		\item[-] \small Para cada critério avaliado, assinale um X na coluna SIM apenas se considerado aprovado. Caso contrário, indique as alterações necessárias na coluna Observação.
	\end{itemize}

\vspace{5pt}

	\noindent\resizebox{\textwidth}{!}{
		\centering
		\begin{tabular}{|X p{6.5cm}|X p{0.7cm}|X p{1.5cm}|X p{0.7cm}|X p{1.5cm}|X p{5.2cm}|}
			%\begin{small}\hhline{*{1}{{|~}*{4}{\arrayrulecolor{shadecolor}|-}|~}
				\hline
				%\multirow{2}{*}{\textbf{Critérios}} &  \textbf{Sim} &  \textbf{Parcial} &  \textbf{Não} &  \textbf{Não se aplica} & \multirow{2}{*}{\textbf{Observação}}\\ \hline{>{\arrayrulecolor{shadecolor}}}
			    \cellcolor{shadecolor} & \multicolumn{4}{c|}{\cellcolor{shadecolor} \textbf{Aprovado}} & \cellcolor{shadecolor} \\ \hhline{*{1}{>{\arrayrulecolor{shadecolor}}-}*{4}{>{\arrayrulecolor{black}}|-}>{\arrayrulecolor{shadecolor}}|->{\arrayrulecolor{black}}}
			    \multirow{-1}{*}{\cellcolor{shadecolor} {\small \textbf{Critérios}}} & \cellcolor{shadecolor} {\small \textbf{Sim}} &  \cellcolor{shadecolor} {\small \textbf{Parcial}} &  \cellcolor{shadecolor} {\small \textbf{Não}} &  \cellcolor{shadecolor} {\small \textbf{Não se aplica}} & \multirow{-1}{*}{\cellcolor{shadecolor} {\small \textbf{Observação}}} \\ \hline
				{\small 1.O trabalho é adequado para um TCC no CCO/SIN (relevância/abrangência)?} & \cellcolor{shadecolor} X & \cellcolor{shadecolor}  & \cellcolor{shadecolor}  & \cellcolor{shadecolor}  & \\ \hline
				{\small 2.O título do trabalho é adequado?} & \cellcolor{shadecolor} X & \cellcolor{shadecolor}  & \cellcolor{shadecolor}  & \cellcolor{shadecolor}  & \\ \hline
				{\small 3.O tema de pesquisa está claramente descrito?} & \cellcolor{shadecolor} X & \cellcolor{shadecolor}  & \cellcolor{shadecolor}  & \cellcolor{shadecolor}  & \\ \hline
				{\small 4.O problema/hipóteses de pesquisa do trabalho está claramente identificado?} & \cellcolor{shadecolor} X & \cellcolor{shadecolor}  & \cellcolor{shadecolor}  & \cellcolor{shadecolor}  & \\ \hline
				{\small 5.A relevância da pesquisa é justificada?} & \cellcolor{shadecolor} X & \cellcolor{shadecolor}  & \cellcolor{shadecolor}  & \cellcolor{shadecolor}  & \\ \hline
				{\small 6.Os objetivos descrevem completa e claramente o que se pretende alcançar neste trabalho?} & \cellcolor{shadecolor} X & \cellcolor{shadecolor}  & \cellcolor{shadecolor}  & \cellcolor{shadecolor}  & \\ \hline
				{\small 7.É definido o método a ser adotado no trabalho? O método condiz com os objetivos e é adequado para um TCC? } & \cellcolor{shadecolor} X & \cellcolor{shadecolor}  & \cellcolor{shadecolor}  & \cellcolor{shadecolor}  & \\ \hline
				{\small 8.Foi definido um cronograma coerente com o método definido (indicando todas as atividades) e com as datas das entregas (p.ex.Projeto I, II, Defesa)?} & \cellcolor{shadecolor} X & \cellcolor{shadecolor}  & \cellcolor{shadecolor}  & \cellcolor{shadecolor}  & \\ \hline
				{\small 9.Foram identificados custos relativos à execução deste trabalho (se houver)? Haverá financiamento para estes custos?} & \cellcolor{shadecolor} X & \cellcolor{shadecolor}  & \cellcolor{shadecolor}  & \cellcolor{shadecolor}  & \\ \hline
				{\small 10.Foram identificados todos os envolvidos neste trabalho?} & \cellcolor{shadecolor} X & \cellcolor{shadecolor}  & \cellcolor{shadecolor}  & \cellcolor{shadecolor}  & \\ \hline
				{\small 11.As formas de comunicação foram definidas (ex: horários para orientação)?} & \cellcolor{shadecolor} X & \cellcolor{shadecolor}  & \cellcolor{shadecolor}  & \cellcolor{shadecolor}  & \\ \hline
				{\small 12.Riscos potenciais que podem causar desvios do plano foram identificados?} & \cellcolor{shadecolor} X & \cellcolor{shadecolor}  & \cellcolor{shadecolor}  & \cellcolor{shadecolor}  & \\ \hline
				{\small 13.Caso o TCC envolva a produção de um software ou outro tipo de produto e seja desenvolvido também como uma atividade	realizada numa empresa ou laboratório, consta da proposta uma declaração (Anexo A) de ciência e concordância com a entrega do código fonte e/ou documentação produzidos? } & \cellcolor{shadecolor} X & \cellcolor{shadecolor}  & \cellcolor{shadecolor}  & \cellcolor{shadecolor}  & \\ \hline

			%\end{small}
		\end{tabular}
	}

\vspace{5pt}

	\noindent\resizebox{\textwidth}{!}{
		\begin{tabular}{| p{3.4cm}| p{2.35cm}| p{1.4cm}| p{5.4cm}|}
				\hline
			    {\tiny \textbf{Avaliação}} &  \multicolumn{1}{l}{\textbf{$\boxtimes$ \tiny Aprovado}}  & \multicolumn{2}{c|}{\textbf{$\Box$ \tiny Não Aprovado}}  \\ \hline \hline

			    {\tiny \textbf{Professor Responsável}} &  {\tiny Prof. Dr. José Luís A. Güntzel} & {\tiny 13/11/2023} & \\ \hline
                {\tiny \textbf{Orientador}} &  {\tiny Me. André Beims Bräscher} & {\tiny 13/11/2023} & \\ \hline

		\end{tabular}
	}


\end{folhadeaprovacao}

% ---

% ---
% Dedicatória
% ---
%\begin{dedicatoria}
%	\vspace*{\fill}
%	\noindent
%	\begin{adjustwidth*}{}{5.5cm}     
%		Este trabalho é dedicado aos meus colegas de classe e aos meus queridos pais.
%	\end{adjustwidth*}
%\end{dedicatoria}
% ---

% ---
% Agradecimentos
% ---
%\begin{agradecimentos}
%	Inserir os agradecimentos aos colaboradores à execução do trabalho. 
	
%	Xxxxxxxxxxxxxxxxxxxxxxxxxxxxxxxxxxxxxxxxxxxxxxxxxxxxxxxxxxxxxxxx. 
%\end{agradecimentos}
% ---

% ---
% Epígrafe
% ---
%\begin{epigrafe}
%	\vspace*{\fill}
%	\begin{flushright}
%		\textit{``Texto da Epígrafe.\\
%			Citação relativa ao tema do trabalho.\\
%			É opcional. A epígrafe pode também aparecer\\
%			na abertura de cada seção ou capítulo.\\
%			Deve ser elaborada de acordo com a NBR 10520.''\\
%			(Autor da epígrafe, ano)}
%	\end{flushright}
%\end{epigrafe}
% ---

% ---
% RESUMOS
% ---

% resumo em português
\setlength{\absparsep}{18pt} % ajusta o espaçamento dos parágrafos do resumo
\begin{resumo}
	\SingleSpacing
	
	%No resumo são ressaltados o objetivo da pesquisa, o método utilizado, as discussões e os resultados com destaque apenas para os pontos principais. O resumo deve ser significativo, composto de uma sequência de frases concisas, afirmativas, e não de uma enumeração de tópicos. Não deve conter citações. Deve usar o verbo na voz ativa e na terceira pessoa do singular. O texto do resumo deve ser digitado, em um único bloco, sem espaço de parágrafo. O espaçamento entre linhas é simples e o tamanho da fonte é 12. Abaixo do resumo, informar as palavras-chave (palavras ou expressões significativas retiradas do texto) ou, termos retirados de thesaurus da área. Deve conter de 150 a 500 palavras. O resumo é elaborado de acordo com a NBR 6028.

    Considerando o aumento significativo observado no consumo e compartilhamento de vídeos nos últimos anos, é notória a demanda por técnicas que permitam armazenamento, transmissão e reprodução desse tipo de dado de forma cada vez mais otimizada.
    Para essa finalidade, a abordagem mais consolidada é a codificação de vídeo híbrida, que tem sido parte essencial da área de codificação de vídeo nas últimas décadas.
    Contudo, estima-se que a complexidade de algoritmos tradicionais híbridos venha aumentando em 10X para um ganho de desempenho de 2X, a cada geração. 
    Nesse contexto, surge a necessidade de buscar modelos alternativos de codificação, como o \ac{DIVC}. 
    O \ac{DIVC} combina o modelo de codificação híbrido com \ac{VFI}, porém tem desempenho variável de acordo com o conteúdo do vídeo.
    Portanto, o objetivo deste trabalho é propor uma técnica para melhorar sua eficiência de codificação a partir da remoção de quadros de forma adaptativa. Para isso, os quadros serão analisados utilizando descritores que apresentem melhor correlação com a eficiência do modelo.
    
    \textbf{Palavras-chave:} Codificação de vídeo. VVC. Eficiência de codificação. Descritores de imagem. VFI. Redes Neurais.
\end{resumo}

% resumo em inglês
%\begin{resumo}[Abstract]
%	\SingleSpacing
%	\begin{otherlanguage*}{english}
%		Resumo traduzido para outros idiomas, neste caso, inglês. Segue o formato do resumo feito na língua vernácula. As palavras-chave traduzidas, versão em língua estrangeira, são colocadas abaixo do texto precedidas pela expressão “Keywords”, separadas por ponto.
		
%		\textbf{Keywords}: Keyword 1. Keyword 2. Keyword 3.
%	\end{otherlanguage*}
%\end{resumo}

%% resumo em francês 
%\begin{resumo}[Résumé]
% \begin{otherlanguage*}{french}
%    Il s'agit d'un résumé en français.
% 
%   \textbf{Mots-clés}: latex. abntex. publication de textes.
% \end{otherlanguage*}
%\end{resumo}
%
%% resumo em espanhol
%\begin{resumo}[Resumen]
% \begin{otherlanguage*}{spanish}
%   Este es el resumen en español.
%  
%   \textbf{Palabras clave}: latex. abntex. publicación de textos.
% \end{otherlanguage*}
%\end{resumo}
%% ---

{%hidelinks
	\hypersetup{hidelinks}
	% ---
	% inserir lista de ilustrações
	% ---
	\pdfbookmark[0]{\listfigurename}{lof}
	% \listoffigures*
	\cleardoublepage
	% ---
	
	% ---
	% inserir lista de quadros
	% ---
	%\pdfbookmark[0]{\listofquadrosname}{loq}
	%\listofquadros*
	%\cleardoublepage
	% ---
	
	% ---
	% inserir lista de tabelas
	% ---
	%\pdfbookmark[0]{\listtablename}{lot}
	%\listoftables*
	%\cleardoublepage
	% ---
	
	% ---
	% inserir lista de abreviaturas e siglas (devem ser declarados no preambulo)
	% ---
	\imprimirlistadesiglas
	% ---
	
	% ---
	% inserir lista de símbolos (devem ser declarados no preambulo)
	% ---
	%\imprimirlistadesimbolos
	% ---
	
	% ---
	% inserir o sumario
	% ---
	\pdfbookmark[0]{\contentsname}{toc}
	\tableofcontents*
	\cleardoublepage
	
}%hidelinks
% ---

% Elementos textuais
\textual

% TODO: Ajustar isso para a entrega final

\chapter{Introdução}

Lesões de pele são áreas anormais ou danificadas da pele, causadas por diversos fatores, como condições ambientais, infecções ou doenças graves, incluindo o câncer de
pele. A identificação precisa da causa é fundamental para o tratamento adequado, porém, devido à semelhança entre os sintomas, diagnosticar corretamente as doenças que
afetam a pele pode ser uma tarefa complexa. Além disso, o prognóstico pode se agravar significativamente se o tratamento não for iniciado cedo o suficiente
\cite{habif2015clinical}.

A importância do cuidado com lesões de pele é evidenciada pelo impacto do câncer de pele. No Brasil, 30\% dos tumores malignos registrados são atribuídos a essa doença
\cite{skin_cancer_in_brazil}. Os tipos mais comuns, o carcinoma basocelular e o carcinoma epidermoide, apresentam baixa letalidade, mas podem deixar sequelas
significativas devido ao tratamento. Já o melanoma, embora menos frequente, é altamente letal e corresponde a cerca de 75\% das mortes causadas pelo câncer de pele
\cite{skin_cancer_screening}. A detecção precoce é essencial para a garantia da efetividade do tratamento, pois a taxa de sobrevida dos pacientes tende a cair ao longo
do avanço da doença \cite{skin_cancer_survival}.

A dificuldade da identificação precoce de lesões de pele tem um componente socioeconômico. Em um estudo feito por \textcite{skin_cancer_socioeconomic}, observa-se que
indivíduos com um grau menor de escolaridade têm um prognóstico pior, pois apresentam tumores em estágios mais avançados. Esta correlação pode indicar que a desigualdade
resulta no atraso do diagnóstico e consequentemente na redução da taxa de sucesso do tratamento. \textcite{santos2020desigualdades} também constata que a saúde dos
brasileiros está associada à renda e classe econômica.

No Brasil, uma parte da atenção primária é feita por \ac{ACS}, que possuem um treinamento básico para atender a população e dar orientações simples de saúde. Esses
são os profissionais que normalmente estão em contato com as regiões mais carentes. A identificação da causa de uma lesão de pele está além das responsabilidades de um
\ac{ACS} \cite{filgueiras2011agente}. Isso pode ser problemático, pois para realizar a triagem de um caso de lesão de pele, se faz necessária a análise de um dermatologista,
que nem sempre está disponível em regiões carentes.

Considerando este cenário, seria útil a existência de um serviço com uma boa usabilidade que analisasse imagens de pacientes e recomendasse uma forma de triagem,
mitigando a carência de dermatologistas. As capacidades de classificar e descrever imagens precisamente tornam \acp{MLLM} bons candidatos para a fundação desse serviço
\cite{mllm_success_rate}.

Neste trabalho, o \ac{LLaMA} 3.2 foi escolhido como o modelo base para o desenvolvimento, pois ele tem o código aberto e também é relevante no estado da arte em tarefas
visuais e textuais \cite{dubey2024llama}. O modelo foi treinado com técnicas de \textit{fine-tuning} sobre o conjunto de dados do \ac{STT/SC} para a classificação e lesões
de pele e a geração de laudos sobre imagens.

Com o objetivo utilizar os recursos computacionais disponíveis eficientemente, o \textit{fine-tuning} foi feito com técnicas baseadas em \ac{PEFT}. Este método permite a
redução do número de parâmetros usados no processo, resultando em uma utilização menor de recursos \cite{peft}. Especificamente, utilizaram-se os métodos \ac{QLoRA} e
\ac{LoRA} para o treinamento do modelo.

\section{Objetivos}

O objetivo principal deste trabalho é treinar um \ac{LLM} multimodal com métodos de \textit{fine-tuning} para a classificação de lesões de pele e geração de laudos,
avaliando o desempenho dos modelos resultantes e comparando as diferentes técnicas de treinamento.

\subsection*{Objetivos Específicos}

\begin{itemize}
    \item Treinar o modelo \ac{LLaMA} 3.2 com as técnicas de \textit{fine-tuning} \ac{QLoRA} e \ac{LoRA};
    \item Comparar os modelos treinados com o \ac{LLaMA} 3.2;
    \item Avaliar os modelos treinados e comparar as diferentes técnicas de \textit{fine-tuning} entre si.
\end{itemize}

\section{Organização do Trabalho}

% TODO: Rever isso no final

O trabalho apresenta inicialmente os conceitos teóricos sobre lesões de pele e \acp{MLLM}, abordando questões sobre a arquitetura e treinamento destes modelos. Logo
em seguida, é feita uma apresentação e discussão sobre trabalhos correlatos. Depois, são apresentados os experimentos, abordando os dados e tecnologias utilizadas,
características dos treinamentos e os resultados. Por fim, é feita a discussão sobre os resultados e possíveis trabalhos futuros são apresentados.

\chapter{Fundamentação Teórica}

Neste capítulo serão discutidos os aspectos relacionados à classificação de imagens de lesões de pele e detecção de melanomas. Além disso, serão explicados os conceitos
de visão computacional com redes neurais, \ac{LLMs} e como estas tecnologias se integram em um \ac{MLLM}. Por fim, será discutido sobre diferentes métodos de
\textit{fine-tuning}.

\section{Lesões de Pele}

A pele é o maior órgão do corpo humano e é responsável por proteger o corpo de agentes microbiológicos, físicos e químicos. Além disso, ela também ajuda na regulação da
temperatura do corpo e, através de receptores cutâneos, proporciona informações sensoriais como o tato \cite{skin}.

Devido à exposição da pele ao ambiente, é mais comum que esse órgão sofra com doenças. As áreas afetadas são consideradas lesões de pele e podem ser usadas para o
diagnóstico da doença \cite{segmentation_skin_lesions}.

\subsection{Câncer de Pele}

O câncer é uma doença caracterizada pela multiplicação de células anormais que podem se espalhar para além do seu tecido de origem, causando tumores e levando
eventualmente à morte \cite{cancer}. O câncer de pele é o tipo mais comum da doença globalmente e é mais frequentemente causado pela exposição prolongada à radiação
ultravioleta \cite{skin_cancer}. Em geral, essa doença afeta mais a pele clara e pode afetar a mesma pessoa mais de uma vez. Uma vez desenvolvido, há um aumento de 35\%
no risco de desenvolvimento de um novo câncer de pele do mesmo tipo em um período de três anos \cite{skin_cancer_zink}.

Existem várias categorias de câncer de pele, elas podem ser agrupadas como \ac{CPNM} e melanoma. \ac{CPNM} podem ser subdivididos em \ac{CBC}, \ac{CEC}, carcinoma de
Merkel e entre outros. Essa categoria é a mais incidente, correspondendo a mais de 90\% dos cânceres diagnosticados e também é a menos fatal. O subtipo \ac{CBC} é o
mais frequente e corresponde a mais de 75\% dos casos de \ac{CPNM} no Brasil \cite{skin_cancer_zink}. O melanoma é o tipo mais fatal e menos incidente da doença. Cerca
de 75\% das mortes por câncer de pele são causadas por melanomas \cite{skin_cancer_screening}.

A doença tem um prognóstico muito melhor quando a detecção e tratamento são feitos cedo o suficiente. Segundo \cite{skin_cancer_survival}, a taxa de sobrevivência ao
melanoma no Brasil é menor que a taxa global, sendo que há uma prevalência maior de casos avançados.

\subsection{Classificação de lesões}

% Características gerais
% classificação automatizada
% Características do câncer de pele

\subsection{Detecção de Melanomas}

% Características do câncer de pele

\section{Descritores de Imagens}

\section{LLMs}

\section{MLLMs}

\section{Fine-tuning}

\chapter{Trabalhos Correlatos} % TODo: Revisar

Nessa seção serão apresentados os trabalhos similares a este para realizar comparações e apresentar melhor o estado da arte nessa linha de pesquisa.

\section{Classificação Automática de Lesões de Pele}

A utilização de \ac{IA} na detecção e classificação de lesões de pele tem se tornado cada vez mais relevante ao longo das últimas décadas. Historicamente, os maiores
avanços nessa área começaram com advento da aprendizagem profunda e das \acp{CNN} \cite{li2019artificial}. A maioria das técnicas tenta classificar câncer de pele, mais
especificamente o melanoma, \ac{CBC} e \ac{CEC}. Porém, as técnicas desenvolvidas no contexto da classificação de câncer de pele são úteis também na classificação de
outras doenças \cite{okuboyejo2018review}.

O trabalho de \textcite{skin_cancer_ai} apresenta que os conjuntos de dados mais comuns são o HAM10000 e os fornecidos pela \ac{ISIC}. Além disso, \acp{CNN} ainda dominam
a área como a principal forma de implementar um sistema de classificação de lesões, atingindo entre 80\% e 99\% de exatidão. Cerca de 48,12\% dos conjuntos de
dados utilizam imagens de dermatoscopia, enquanto 33,33\% deles usam imagens macroscópicas, como as de aproximação e panorâmicas mencionadas na
\autoref{sec:skin_lesion_images}. Os outros 18,52\% se tratam de conjuntos com diversas modalidades de imagens, como imagens de ultrasonografia, multiespectrais e outras.

\section{Classificação com ViTs}

Vários modelos de classificação de lesões baseados em \acp{ViT} foram propostos desde a introdução dessa arquitetura. A pesquisa de \textcite{khan2023identifying}
apresenta o cenário da utilização de \acp{ViT} nessa área. Muitas implementações seguem um modelo híbrido de \ac{ViT} e \ac{CNN}, combinando as capacidades das duas
arquiteturas.

Uma desvantagem de \acp{CNN} é a falta de entendimento de relações espaciais de longa distância em imagens de lesões de pele. Entretanto, \acp{ViT} conseguem resolver
esse problema, capturando as relações espaciais através do mecanismo de atenção. Mas, devido ao processo de divisão da imagem em seções de baixa resolução, \acp{ViT}
acabam tendo um desempenho pior, pois há uma perda de informações sobre detalhes mais finos. Nesse contexto, a utilização de modelos híbridos resolve esses problemas.
Uma dessas arquiteturas híbridas é a \textit{TransUNet}, combinando transformadores e \textit{U-Nets}, que conforme evidenciado por \textcite{gulzar2022skin}, consegue
atingir uma exatidão de 92,11\% na classificação de lesões de pele com o conjunto de dados \ac{ISIC}-2018.

\section{Classificação com MLLMs}

\acp{MLLM} estão sendo aplicados em contexto de dermatologia nos últimos anos. O \ac{GPT}-4V e o \ac{LLaVA} são dois modelos proeminentes na resolução de tarefas
visuais. Em um estudo de \textcite{cirone2024assessing}, é analisada a eficácia do uso desses modelos na identificação de melanomas em diferentes tons de pele. Os
testes foram feitos com os conjuntos de dados MClass-D, Dermnet NZ e imagens de livros de dermatologia, contendo imagens macroscópicas com resolução de
\begin{math}900 \times 1600\end{math} pixels. Para cada imagem foram feitas 20 perguntas em relação a características da imagem, sendo que cada modelo foi testado com 3
imagens, resultando num total de 60 pares de pergunta e imagem por modelo. As imagens também tiveram suas cores modificadas para avaliar o impacto da coloração na
identificação de melanomas.

% TODO: Comentar sobre o modelo do Marques (http://sibgrapi.sid.inpe.br/col/sid.inpe.br/sibgrapi/2024/08.28.22.28/doc/Marques-125.pdf) e o SkinDiseaseChat

No fim, o \ac{GPT}-4V atingiu uma exatidão de 85\% e o \ac{LLaVA} atingiu apenas 45\%. O modelo \ac{LLaVA} não conseguiu identificar melanomas corretamente quando as
imagens tinham suas cores modificadas e também não conseguiu identificar detalhes como ulcerações ou sangramentos. Um detalhe importante é que ambos os modelos são
generalistas e não são treinados com um foco na classificação de lesões de pele.

O \textit{benchmark} OmniMedVQA de \textcite{hu2024omnimedvqa} traz dados sobre o desempenho de \acp{MLLM} em diferentes tarefas visuais da medicina. Em particular, os
testes apresentam as pontuações de diversos modelos na área de dermatologia, como pode ser visto na \autoref{tab:omnimedvqa_dermatology_results}.

\begin{table}[ht]
    \caption{\small Pontuação no \textit{benchmark} OmniMedVQA em dermatologia. A primeira seção de linhas contém modelos generalistas e a outra contém modelos especialidados
        na área médica.}
    \centering
    \begin{tabular}{l|c}
        \hline
        Modelo                  & Pontuação em dermatologia \\ \hline
        InstructBLIP            & 61,86                     \\
        \ac{LLaMA}\_Adapter\_v2 & 51,43                     \\
        \ac{LLaVA}              & 49,67                     \\
        PGTrans                 & 44,66                     \\
        Otter                   & 42,66                     \\
        BLIP-2                  & 41,07                     \\
        Mini\ac{GPT}-4          & 40,09                     \\
        mPLUG-Owl               & 35,98                     \\ \hline
        \ac{LLaVA}-Med          & 44,90                     \\
        RadFM                   & 39,03                     \\
        Med-Flamingo            & 32,33                     \\
        MedVInT                 & 29,13                     \\ \hline
    \end{tabular}
    \label{tab:omnimedvqa_dermatology_results}
    \fonte{\textcite{hu2024omnimedvqa}}
\end{table}

Nas subseções abaixo, serão apresentados alguns modelos que focam em problemas relacionados à classificação de lesões de pele.

\subsection{SkinGPT-4}

Esse modelo foi proposto e desenvolvido por \textcite{zhou2023skingpt} e se baseia no Mini\ac{GPT}-4, que é um \ac{MLLM} composto por um \ac{ViT} com um \textit{Q-Former}
pré-treinado e o \ac{LLM} Vicuna, que por sua vez é baseado no \ac{LLaMA}. Nesse estudo, o Mini\ac{GPT}-4 passou por um processo de \textit{fine-tuning} de duas etapas
que familiarizou o modelo com imagens de lesões de pele e depois ajustou suas saídas para um formato mais próximo de um diagnóstico médico. As imagens vieram de um
conjunto de dados particular, do SKINCON e do Dermnet, sendo que 3886 imagens foram usadas na primeira etapa de treinamento e 49043 na segunda.

A verificação do desempenho do modelo foi feito com base em 150 diagnósticos de casos reais realizados pelo Skin\ac{GPT}-4. Esses diagnósticos foram avaliados por
dermatologistas, classificando 78,76\% dos diagnósticos como corretos.

\subsection{LLaVA-Med}

\subsection{MpoxVLM}

\subsection{Comparação entre MLLMs}

% TODO: Atualizar isso no futuro
\begin{table}[ht]
    \caption{\small Comparação.}
    \centering
    \begin{tabular}{l|cc}
        \hline
        Modelo                & MLLM base ou arquitetura & Treinamento          \\ \hline
        SkinGPT-4             & MiniGPT-4                & Fine-tuning completo \\
        LLaVA-Med             & LLaVA                    & Fine-tuning completo \\
        MpoxVLM               & CLIP e LLaMA-2           & LoRA                 \\
        Modelo deste trabalho & LLaMA-3.2                & QLoRA e LoRA         \\ \hline
    \end{tabular}
    \label{tab:mllm_comparison}
    \fonte{Autoria própria.}
\end{table}
\chapter{Desenvolvimento Parcial}

\section{Tecnologias Utilizadas}

\section{Conjunto de Dados}

\section{Experimentos com \textit{Fine-tuning}}

\section{Testes}

\section{Análise dos resultados parciais}

%\phantompart
\chapter{Conclusão}

Neste trabalho, apresentou-se a fundamentação teórica sobre \acp{MLLM} e suas aplicações na classificação de lesões de pele. Além disso, os experimentos realizados
forneceram dados relevantes para a discussão sobre metodologias de \textit{fine-tuning}.

Os resultados apresentados indicam que o modelo \ac{LLaMA}-3.2 pode ser treinado eficientemente através do \textit{fine-tuning} com \ac{QLoRA} e \ac{LoRA} para atingir
desempenhos satisfatórios na classificação de lesões de pele. Constatou-se que o \ac{QLoRA} leva a um uso consideravelmente menor de memória durante o treinamento, o
que é vantajoso para o desenvolvimento do modelo. Porém, algumas características dos experimentos não foram avaliadas, como, por exemplo, o \textit{overfitting}. % TODO: Talvez colocar a tradução no rodapé

\section{Próxima Etapa e Planejamento}

Na próxima etapa, planeja-se utilizar o conjunto de dados do \ac{STT/SC}, permitindo que o \textit{fine-tuning} seja feito com laudos detalhados e em português. Além
disso, esse conjunto de dados possui uma diversidade maior de lesões e não foca somente no câncer de pele.

O refinamento dos principais hiperparâmetros de \textit{fine-tuning} é também um objetivo para a próxima etapa. Um estudo mais aprofundado da teoria por trás dessas
configurações deve ser feito para garantir a qualidade dos treinamentos. Ferramentas como W\&B Sweeps e Optuna podem ser utilizadas para a otimização dos hiperparâmetros
experimentalmente \cite{wab_sweeps, optuna_2019}.

Além disso, é necessário desenvolver um mecanismo mais adequado de testes para modelos, de modo a avaliar respostas mais complexas. Pretende-se utilizar um \ac{LLM}
para esta análise, assim será possível extrair informações relevantes de uma grande quantidade de respostas de forma automática.

Por fim, planeja-se realizar o \textit{fine-tuning} com a variante do \ac{LLaMA}-3.2 com 90 bilhões de parâmetros, comparando os modelos resultantes desse treinamento
com os baseados na variante de 11 bilhões de parâmetros.

\subsection{Cronograma}

A \autoref{tab:tcc_development} apresenta o plano de desenvolvimento do \ac{TCC} com o trabalho realizado até o momento e as atividades que ainda serão realizadas.

\begin{table}[]
    \resizebox{\columnwidth}{!}{%
        \begin{tabular}{|l|lllll|}
            \hline
            \rowcolor[HTML]{EFEFEF}
            \multicolumn{1}{|c|}{\cellcolor[HTML]{EFEFEF}}                                                                                                             & \multicolumn{5}{c|}{\cellcolor[HTML]{EFEFEF}Meses}                                                                                                                                                                                                                                                                         \\ \cline{2-6}
            \rowcolor[HTML]{EFEFEF}
            \multicolumn{1}{|c|}{\multirow{-2}{*}{\cellcolor[HTML]{EFEFEF}Etapas}}                                                                                     & \multicolumn{1}{c}{\cellcolor[HTML]{EFEFEF}03}                       & 04                                                                   & 05                                                                   & 06                                                                   & 07                             \\ \hline
            \begin{tabular}[c]{@{}l@{}}Tratamento do conjunto de dados do STT/SC\\ e desenvolvimento da base de dados\end{tabular}                                     & \multicolumn{1}{c|}{\cellcolor[HTML]{C0C0C0}{\color[HTML]{241F31} }} & \multicolumn{1}{l|}{}                                                & \multicolumn{1}{l|}{}                                                & \multicolumn{1}{l|}{}                                                &                                \\ \hline
            Desenvolvimento dos testes                                                                                                                                 & \multicolumn{1}{l|}{}                                                & \multicolumn{1}{c|}{\cellcolor[HTML]{C0C0C0}{\color[HTML]{241F31} }} & \multicolumn{1}{l|}{}                                                & \multicolumn{1}{l|}{}                                                &                                \\ \hline
            \begin{tabular}[c]{@{}l@{}}Fine-tuning do LLaMA-3.2 com 11 e 90 bilhões de\\ parâmeteros com hiperparâmetros adequados e\\ diferentes métodos\end{tabular} & \multicolumn{1}{l|}{}                                                & \multicolumn{1}{c|}{\cellcolor[HTML]{C0C0C0}{\color[HTML]{241F31} }} & \multicolumn{1}{l|}{\cellcolor[HTML]{C0C0C0}}                        & \multicolumn{1}{l|}{}                                                &                                \\ \hline
            Aplicação dos testes e análise dos resultados                                                                                                              & \multicolumn{1}{c|}{}                                                & \multicolumn{1}{l|}{}                                                & \multicolumn{1}{c|}{\cellcolor[HTML]{C0C0C0}{\color[HTML]{241F31} }} & \multicolumn{1}{c|}{}                                                &                                \\ \hline
            Entrega do Relatório do TCC II                                                                                                                             & \multicolumn{1}{l|}{}                                                & \multicolumn{1}{l|}{}                                                & \multicolumn{1}{l|}{}                                                & \multicolumn{1}{c|}{\cellcolor[HTML]{C0C0C0}{\color[HTML]{241F31} }} & \multicolumn{1}{c|}{\textbf{}} \\ \hline
            Defesa pública                                                                                                                                             & \multicolumn{1}{l|}{}                                                & \multicolumn{1}{l|}{}                                                & \multicolumn{1}{l|}{}                                                & \multicolumn{1}{l|}{\cellcolor[HTML]{C0C0C0}}                        & \multicolumn{1}{c|}{\textbf{}} \\ \hline
            Ajustes finais no relatório do TCC                                                                                                                         & \multicolumn{1}{l|}{}                                                & \multicolumn{1}{l|}{}                                                & \multicolumn{1}{l|}{}                                                & \multicolumn{1}{l|}{}                                                & \cellcolor[HTML]{C0C0C0}       \\ \hline
        \end{tabular}%
    }
\end{table}

% Elementos pós-textuais
\postextual

% Referências bibliográficas
\begingroup
\printbibliography[title=REFERÊNCIAS]
\endgroup

% Apêndices
%\begin{apendicesenv}
%	\partapendices* 
%	\input{aftertext/apendice_a}
%\end{apendicesenv}

\end{document}
