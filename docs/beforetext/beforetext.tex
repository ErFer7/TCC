% Capa
\imprimircapa

% Folha de rosto
\imprimirfolhaderosto*

\begin{fichacatalografica}
	\includepdf{beforetext/Ficha_Catalografica.pdf}
\end{fichacatalografica}

% Folha de aprovação
\begin{folhadeaprovacao}
	\begin{center}
		{\imprimirautor}

		\begin{center}
			\ABNTEXchapterfont\bfseries\MakeUppercase{\imprimirtitulo}\ifnotempty{\imprimirsubtitulo}{: \imprimirsubtitulo}
		\end{center}

		\begin{minipage}{\textwidth}
			Este Trabalho de Conclusão de Curso foi julgado adequado para obtenção do Título de \imprimirformacao,
			e foi aprovado em sua forma final pelo Curso de Ciência da Computação.
		\end{minipage}%
	\end{center}

	\begin{center}
		\imprimirlocal, 15 de julho de 2025.
	\end{center}

	\assinatura{
		\textbf{\imprimircoordenador} \\
		Coordenação do Curso
	}

	\begin{center}
		\vspace{1cm}
		\textbf{Banca Examinadora:}
	\end{center}

	\vspace{1cm}
	\assinatura{
		\textbf{\imprimirorientador} \\ \imprimirorientadorRotulo
	}

	% \imprimircoorientador{
	% 	\assinatura{
	% 		\textbf{\imprimircoorientador} \\ \imprimircoorientadorRotulo \\
	% 		\imprimirinstituicao~--~\imprimirinstituicaosigla
	% 	}
	% }

	\vspace{1cm}
	\assinatura{
		\textbf{Prof. Convidado 1, Dr.} \\
		Instituição 1 -- Sigla 1
	}

	\vspace{1cm}
	\assinatura{
		\textbf{Prof. Convidado 2, Dr.} \\
		Instituição 2 -- Sigla 2
	}

	\begin{center}
		\vfill
		{
			\imprimirlocal\par
			\imprimirano\par
		}
	\end{center}
\end{folhadeaprovacao}

% \include{beforetext/thanks}

% Resumo em português
\setlength{\absparsep}{18pt}
\begin{resumo}
	\SingleSpacing

	Lesões de pele podem ser um indicativo de diversas doenças, incluindo doenças graves como o câncer de pele. A detecção precoce dessas lesões é fundamental para o
	tratamento e cura da doença. Porém, o diagnóstico e classificação de uma lesão de pele é normalmente feita por profissionais especializados em hospitais ou clínicas.
	Isto pode levar a um atraso no diagnóstico pela falta de acesso ou procura pelo atendimento médico.

	Considerando este cenário, tecnologias como \ac{MLLMs} podem ser úteis. Estes modelos podem identificar lesões de pele com base em imagens e prover um pré-diagnóstico
	que pode alertar o portador da lesão sobre a necessidade de procurar atendimento médico. O modelo \ac{LLaVa} é um bom candidato para esta aplicação, pois consegue
	descrever imagens e pode ser adaptado para propósitos específicos através de \textit{fine-tuning}.

	Neste trabalho, propõe-se a adaptação do \ac{LLaVa} com diferentes técnicas de \textit{fine-tuning}, comparando-as entre si, para classificar lesões de pele com uma
	precisão aceitável.

	\textbf{Palavras-chave}: Lesões de pele. MLLM. LLaVa. Fine tuning.
\end{resumo}

% Resumo em inglês
\begin{resumo}[Abstract]
	\SingleSpacing
	\begin{otherlanguage*}{english}

		Skin lesions can indicate several diseases, including serious diseases such as skin cancer. Early detection of these lesions is essential for the treatment
		and cure of the disease. However, the diagnosis and classification of a skin lesion is normally carried out by specialized professionals in hospitals or clinics.
		This can lead to a delay in diagnosis due to a lack of access to or demand for medical care.

		Considering this scenario, technologies like \ac{MLLMs} can be useful. These models can identify skin lesions based on images and provide pre-diagnosis which
		can alert the person suffering from the lesion about the need to seek medical attention. The \ac{LLaVa} model  is a good candidate for this application, as it can
		describe images and be adapted for specific purposes through fine-tuning.

		In this work, we propose the adaptation of \ac{LLaVa} with different fine-tuning techniques, comparing them to each other, to classify skin lesions with an
		acceptable accuracy.

		\textbf{Keywords}: Skin lesions. MLLM. LLaVa. Fine tuning.
	\end{otherlanguage*}
\end{resumo}

{
\hypersetup{hidelinks}

% Lista de siglas
\imprimirlistadesiglas

% Sumário
\pdfbookmark[0]{\contentsname}{toc}
\tableofcontents*
\cleardoublepage
}
