\chapter{Fundamentação Teórica}

Neste capítulo serão discutidos os aspectos relacionados à classificação de imagens de lesões de pele e detecção de melanomas. Além disso, serão explicados os conceitos
de visão computacional com redes neurais, \ac{LLMs} e como estas tecnologias se integram em um \ac{MLLM}. Por fim, será discutido sobre diferentes métodos de
\textit{fine-tuning}.

\section{Lesões de Pele}

A pele é o maior órgão do corpo humano e é responsável por proteger o corpo de agentes microbiológicos, físicos e químicos. Além disso, ela também ajuda na regulação da
temperatura do corpo e, através de receptores cutâneos, proporciona informações sensoriais como o tato \cite{skin}.

Devido à exposição da pele ao ambiente, é mais comum que esse órgão sofra com doenças. As áreas afetadas são consideradas lesões de pele e podem ser usadas para
diagnósticos \cite{segmentation_skin_lesions}.

\subsection{Câncer de Pele}

O câncer é uma doença caracterizada pela multiplicação de células anormais que podem se espalhar para além do seu tecido de origem, causando tumores e levando
eventualmente à morte \cite{cancer}. O câncer de pele é o tipo mais comum da doença globalmente e é mais frequentemente causado pela exposição prolongada à radiação
ultravioleta \cite{skin_cancer}. Em geral, essa doença afeta mais a pele clara e pode afetar a mesma pessoa mais de uma vez. Uma vez desenvolvido, há um aumento de 35\%
no risco de desenvolvimento de um novo câncer de pele do mesmo tipo em um período de três anos \cite{skin_cancer_zink}.

Existem várias categorias de câncer de pele, elas podem ser agrupadas como \ac{CPNM} e melanoma. \ac{CPNM} podem ser subdivididos em \ac{CBC}, \ac{CEC}, carcinoma de
Merkel e entre outros. Essa categoria é a mais incidente, correspondendo a mais de 90\% dos cânceres diagnosticados e também é a menos fatal. O subtipo \ac{CBC} é o
mais frequente e corresponde a mais de 75\% dos casos de \ac{CPNM} no Brasil \cite{skin_cancer_zink}. O melanoma é o tipo mais fatal e menos incidente da doença. Cerca
de 75\% das mortes por câncer de pele são causadas por melanomas \cite{skin_cancer_screening}.

A doença tem um prognóstico muito melhor quando a detecção e tratamento são feitos cedo o suficiente. Segundo \cite{skin_cancer_survival}, a taxa de sobrevivência ao
melanoma no Brasil é menor que a taxa global, sendo que há uma prevalência maior de casos avançados.

\subsection{Imagens de Lesões de Pele}

As imagens para exames dermatológicos podem ser agrupadas em diferentes categorias. Alguns exemplos baseados em procedimentos recomendados por \cite{fotos_dermatologia}
são:

\begin{itemize}
    \item \textbf{Dermatoscopia}: Essas imagens são obtidas com um equipamento especializado, o dermatoscópio. Esse método permite que um diagnóstico mais preciso seja
          feito, sendo melhor que o olho nu na detecção de melanomas \cite{dermatoscopy}.
    \item \textbf{Foto de aproximação com régua}: Esse tipo de imagem é obtida com uma fotografia feita a 30 centímetros da lesão, sem utilizar \textit{zoom}.
          Nessas imagens, etiquetas são colocadas próximas à lesão para auxiliar na determinação do seu tamanho.
    \item \textbf{Foto panorâmica}: Nesse método são registradas imagens de regiões do corpo, como da cabeça, tronco, braços e pernas.
\end{itemize}

\subsection{Detecção e Diagnóstico de Câncer de Pele}

O câncer de pele pode ser identificado e classificado através dos sintomas causados. Características como o tamanho da lesão, variação da cor, irregularidade do formato,
progresso ao longo do tempo e local do corpo em que a lesão se encontra são fundamentais para o diagnóstico da doença \cite{recognizing_skin_cancer}.

A análise destes sintomas é normalmente um processo visual realizado por um dermatologista, sendo que técnicas como a dermatoscopia e teledermatologia podem ser usadas.
Outros métodos não invasivos incluem, por exemplo, a análise intracutânea espectrofotométrica, ultrasonografia de alta frequência e microscopia confocal de refletância.
É possível também diagnosticar a doença através de métodos invasivos como a biópsia. Além destes métodos, há também a utilização de \ac{IA}. Soluções com \ac{IA}
utilizam imagens para a classificação da doença e podem ter uma precisão equiparável ou até maior a de dermatologistas \cite{recognizing_skin_cancer, skin_cancer_ai}.

\section{LLMs}

\section{Visão Computacional}

\section{MLLMs}

\section{Fine-tuning}
