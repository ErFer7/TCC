\chapter{Fundamentação Teórica}

Neste capítulo serão evidenciados os conceitos fundamentais para o entendimento do trabalho. Serão discutidos os aspectos relacionados à classificação de imagens de
lesões de pele e sobre o funcionamento de \ac{MLLMs}

\section{Lesões de Pele}

A pele é o maior órgão do corpo humano e é responsável por proteger o corpo de agentes microbiológicos, físicos e químicos. Além disso, ela também ajuda na regulação da
temperatura do corpo e, através de receptores cutâneos, proporciona informações sensoriais como o tato \cite{skin}.

Devido à exposição da pele ao ambiente, é mais comum que esse órgão sofra com doenças. As áreas afetadas são consideradas lesões de pele e podem ser usadas para o
diagnóstico da doença \cite{segmentation_skin_lesions}.

\subsection{Câncer de Pele}

\subsection{Classificação de lesões}

\subsection{Detecção de Melanomas}

\section{Descritores de Imagens}

\section{LLMs}

\section{MLLMs}

\section{Fine-tuning}
