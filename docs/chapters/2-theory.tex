\chapter{Fundamentação Teórica}

Neste capítulo serão discutidos os aspectos relacionados à classificação de imagens de lesões de pele e detecção de melanomas. Além disso, serão explicados os conceitos
de visão computacional com redes neurais, \ac{LLMs} e como estas tecnologias se integram em um \ac{MLLM}. Por fim, será discutido sobre diferentes métodos de
\textit{fine-tuning}.

\section{Lesões de Pele}

A pele é o maior órgão do corpo humano e é responsável por proteger o corpo de agentes microbiológicos, físicos e químicos. Além disso, ela também ajuda na regulação da
temperatura do corpo e, através de receptores cutâneos, proporciona informações sensoriais como o tato \cite{skin}.

Devido à exposição da pele ao ambiente, é mais comum que esse órgão sofra com doenças. As áreas afetadas são consideradas lesões de pele e podem ser usadas para o
diagnóstico da doença \cite{segmentation_skin_lesions}.

\subsection{Câncer de Pele}

O câncer é uma doença caracterizada pela multiplicação de células anormais que podem se espalhar para além do seu tecido de origem, causando tumores e levando
eventualmente à morte \cite{cancer}. O câncer de pele é o tipo mais comum da doença globalmente e é mais frequentemente causado pela exposição prolongada à radiação
ultravioleta \cite{skin_cancer}. Em geral, essa doença afeta mais a pele clara e pode afetar a mesma pessoa mais de uma vez. Uma vez desenvolvido, há um aumento de 35\%
no risco de desenvolvimento de um novo câncer de pele do mesmo tipo em um período de três anos \cite{skin_cancer_zink}.

Existem várias categorias de câncer de pele, elas podem ser agrupadas como \ac{CPNM} e melanoma. \ac{CPNM} podem ser subdivididos em \ac{CBC}, \ac{CEC}, carcinoma de
Merkel e entre outros. Essa categoria é a mais incidente, correspondendo a mais de 90\% dos cânceres diagnosticados e também é a menos fatal. O subtipo \ac{CBC} é o
mais frequente e corresponde a mais de 75\% dos casos de \ac{CPNM} no Brasil \cite{skin_cancer_zink}. O melanoma é o tipo mais fatal e menos incidente da doença. Cerca
de 75\% das mortes por câncer de pele são causadas por melanomas \cite{skin_cancer_screening}.

A doença tem um prognóstico muito melhor quando a detecção e tratamento são feitos cedo o suficiente. Segundo \cite{skin_cancer_survival}, a taxa de sobrevivência ao
melanoma no Brasil é menor que a taxa global, sendo que há uma prevalência maior de casos avançados.

\subsection{Classificação de lesões}

% Características gerais
% classificação automatizada
% Características do câncer de pele

\subsection{Detecção de Melanomas}

% Características do câncer de pele

\section{Descritores de Imagens}

\section{LLMs}

\section{MLLMs}

\section{Fine-tuning}
