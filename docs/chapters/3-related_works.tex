\chapter{Trabalhos Correlatos}

Nessa seção serão apresentados os trabalhos similares a este para realizar comparações e apresentar melhor o estado da arte nessa linha de pesquisa.

\section{Classificação Automática de Lesões de Pele}

A utilização de \ac{IA} na detecção e classificação de lesões de pele tem se tornado cada vez mais relevante ao longo das últimas décadas. Historicamente, os maiores
avanços nessa área começaram com advento da aprendizagem profunda e das \acp{CNN} \cite{li2019artificial}. A maioria das técnicas tenta classificar câncer de pele, mais
especificamente o melanoma, \ac{CBC} e \ac{CEC}. Porém, as técnicas desenvolvidas no contexto da classificação de câncer de pele são úteis também na classificação de
outras doenças \cite{okuboyejo2018review}.

O trabalho de \textcite{skin_cancer_ai} apresenta que os conjuntos de dados mais comuns são o HAM10000 e os fornecidos pela \ac{ISIC}. Além disso, \acp{CNN} ainda dominam
a área como a principal forma de implementar um sistema de classificação de lesões, atingindo entre 80\% e 99\% de exatidão. Cerca de 48,12\% dos conjuntos de
dados utilizam imagens de dermatoscopia, enquanto 33,33\% deles usam imagens macroscópicas, como as de aproximação e panorâmicas mencionadas na
\autoref{sec:skin_lesion_images}. Os outros 18,52\% se tratam de conjuntos com diversas modalidades de imagens, como imagens de ultrasonografia, multiespectrais e outras.

\section{Classificação com ViTs}

Vários modelos de classificação de lesões baseados em \acp{ViT} foram propostos desde a introdução dessa arquitetura. A pesquisa de \textcite{khan2023identifying}
apresenta o cenário da utilização de \acp{ViT} nessa área. Muitas implementações seguem um modelo híbrido de \ac{ViT} e \ac{CNN}, combinando as capacidades das duas
arquiteturas.

Uma desvantagem de \acp{CNN} é a falta de entendimento de relações espaciais de longa distância em imagens de lesões de pele. Entretanto, \acp{ViT} conseguem resolver
esse problema, capturando as relações espaciais através do mecanismo de atenção. Mas, devido ao processo de divisão da imagem em seções de baixa resolução, \acp{ViT}
acabam tendo um desempenho pior, pois há uma perda de informações sobre detalhes mais finos. Nesse contexto, a utilização de modelos híbridos resolve esses problemas.
Uma dessas arquiteturas híbridas é a \textit{TransUNet}, combinando transformadores e \textit{U-Nets}, que conforme evidenciado por \textcite{gulzar2022skin}, consegue
atingir uma exatidão de 92,11\% na classificação de lesões de pele com o conjunto de dados \ac{ISIC}-2018.

\section{Classificação com MLLMs}