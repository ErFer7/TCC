\chapter{Conclusão}

Neste trabalho, apresentou-se a fundamentação teórica sobre \acp{MLLM} e suas aplicações na classificação de lesões de pele. Além disso, os experimentos realizados
forneceram dados relevantes para a discussão sobre metodologias de \textit{fine-tuning}.

Os resultados apresentados indicam que o modelo \ac{LLaMA}-3.2 pode ser treinado eficientemente através do \textit{fine-tuning} com \ac{QLoRA} e \ac{LoRA} para atingir
desempenhos satisfatórios na classificação de lesões de pele. Constatou-se que o \ac{QLoRA} leva a um uso consideravelmente menor de memória durante o treinamento, o
que é vantajoso para o desenvolvimento do modelo. Porém, algumas características dos experimentos não foram avaliadas, como, por exemplo, o \textit{overfitting}. % TODO: Talvez colocar a tradução no rodapé

\section{Próxima Etapa e Planejamento}

Na próxima etapa, planeja-se utilizar o conjunto de dados do \ac{STT/SC}, permitindo que o \textit{fine-tuning} seja feito com laudos detalhados e em português. Além
disso, esse conjunto de dados possui uma diversidade maior de lesões e não foca somente no câncer de pele.

O refinamento dos principais hiperparâmetros de \textit{fine-tuning} é também um objetivo para a próxima etapa. Um estudo mais aprofundado da teoria por trás dessas
configurações deve ser feito para garantir a qualidade dos treinamentos. Ferramentas como W\&B Sweeps e Optuna podem ser utilizadas para a otimização dos hiperparâmetros
experimentalmente \cite{wab_sweeps, optuna_2019}.

Além disso, é necessário desenvolver um mecanismo mais adequado de testes para modelos, de modo a avaliar respostas mais complexas. Pretende-se utilizar um \ac{LLM}
para esta análise, assim será possível extrair informações relevantes de uma grande quantidade de respostas de forma automática.

Por fim, planeja-se realizar o \textit{fine-tuning} com a variante do \ac{LLaMA}-3.2 com 90 bilhões de parâmetros, comparando os modelos resultantes desse treinamento
com os baseados na variante de 11 bilhões de parâmetros.

\subsection{Cronograma}

A \autoref{tab:tcc_development} apresenta o plano de desenvolvimento do \ac{TCC} com o trabalho realizado até o momento e as atividades que ainda serão realizadas.

\definecolor{Silver}{rgb}{0.752,0.752,0.752}
\definecolor{Green}{rgb}{0.639,0.941,0.717}
\noindent \begin{table}[ht]
    \caption{\small Cronograma de desenvolvimento do \ac{TCC}. As atividades em verde já foram realizadas, enquanto as atividades em cinza ainda deverão ser concluídas.}
    \centering
    \begin{tblr}{
        width = \linewidth,
        colspec = {Q[371]Q[37]Q[46]Q[38]Q[40]Q[44]Q[44]Q[44]Q[40]Q[46]Q[40]Q[46]Q[44]Q[37]},
        row{1} = {Silver,c},
        row{2} = {Silver},
        cell{1}{1} = {r=2}{},
        cell{1}{2} = {c=13}{0.545\linewidth},
        cell{2}{2} = {c},
        cell{2}{3} = {c},
        cell{2}{4} = {c},
        cell{2}{5} = {c},
        cell{3}{2} = {Green},
        cell{3}{3} = {Green},
        cell{3}{4} = {Green},
        cell{3}{5} = {Green},
        cell{4}{4} = {Green},
        cell{4}{5} = {Green},
        cell{4}{6} = {Green},
        cell{5}{6} = {Green},
        cell{6}{6} = {Green},
        cell{6}{7} = {Green},
        cell{7}{7} = {Green},
        cell{8}{8} = {Silver},
        cell{8}{9} = {Silver},
        cell{9}{9} = {Silver},
        cell{9}{10} = {Silver},
        cell{10}{10} = {Silver},
        cell{10}{11} = {Silver},
        cell{11}{11} = {Silver},
        cell{11}{12} = {Silver},
        cell{12}{12} = {Silver},
        cell{12}{13} = {Silver},
        cell{13}{13} = {Silver},
        cell{14}{14} = {Silver},
        vlines,
        hline{1,3-14} = {-}{},
        hline{2} = {2-15}{},
        hline{12} = {1-14}{},
        hline{15} = {1-14}{}
            }
        \textbf{Etapas}                                                                                                              & \textbf{Meses} &             &             &             &             &             &             &             &             &             &             &             &             \\
                                                                                                                                     & \textbf{07}    & \textbf{08} & \textbf{09} & \textbf{10} & \textbf{11} & \textbf{12} & \textbf{01} & \textbf{02} & \textbf{03} & \textbf{04} & \textbf{05} & \textbf{06} & \textbf{07} \\
        Estudo da fundamentação teórica dos \acp{MLLM}                                                                               &                &             &             &             &             &             &             &             &             &             &             &             &             \\
        Estudo da utilização de métodos de \textit{fine-tuning} baseados em \ac{PEFT}                                                &                &             &             &             &             &             &             &             &             &             &             &             &             \\
        Pesquisa dos trabalhos correlatos                                                                                            &                &             &             &             &             &             &             &             &             &             &             &             &             \\
        \textit{Fine-tuning} inicial do \ac{LLaMA}-3.2 com \ac{QLoRA} e \ac{LoRA}                                                    &                &             &             &             &             &             &             &             &             &             &             &             &             \\
        Entrega do relatório do \ac{TCC} I                                                                                           &                &             &             &             &             &             &             &             &             &             &             &             &             \\
        Tratamento do conjunto de dados do \ac{STT/SC}                                                                               &                &             &             &             &             &             &             &             &             &             &             &             &             \\
        Desenvolvimento dos testes                                                                                                   &                &             &             &             &             &             &             &             &             &             &             &             &             \\
        \textit{Fine-tuning} do \ac{LLaMA}-3.2 com 11 e 90 bilhões de parâmetros com hiperparâmetros adequados e diferentes métodos &                &             &             &             &             &             &             &             &             &             &             &             &             \\
        Aplicação dos testes e análise dos resultados                                                                                &                &             &             &             &             &             &             &             &             &             &             &             &             \\
        Entrega do relatório do \ac{TCC} II                                                                                          &                &             &             &             &             &             &             &             &             &             &             &             &             \\
        Defesa pública                                                                                                               &                &             &             &             &             &             &             &             &             &             &             &             &             \\
        Ajustes finais no relatório do \ac{TCC}                                                                                      &                &             &             &             &             &             &             &             &             &             &             &             &
    \end{tblr}
    \label{tab:tcc_development}
    \fonte{Autoria própria.}
\end{table}
