% TODO: Ajustar isso para a entrega final

\chapter{Introdução}

Lesões de pele são áreas anormais ou danificadas da pele, causadas por diversos fatores, como condições ambientais,
infecções ou doenças graves, incluindo o câncer de pele. A identificação precisa da causa é fundamental para o
tratamento adequado, porém, devido à semelhança entre os sintomas, diagnosticar corretamente as doenças que afetam a
pele pode ser uma tarefa complexa. Além disso, o prognóstico pode se agravar significativamente se o tratamento não for
iniciado cedo o suficiente \cite{habif2015clinical}.

A importância do cuidado com lesões de pele é evidenciada pelo impacto do câncer de pele. No Brasil, 30\% dos tumores
malignos registrados são atribuídos a essa doença \cite{skin_cancer_in_brazil}. Os tipos mais comuns, o carcinoma
basocelular e o carcinoma epidermoide, apresentam baixa letalidade, mas podem deixar sequelas significativas devido ao
tratamento. Já o melanoma, embora menos frequente, é altamente letal e corresponde a cerca de 75\% das mortes causadas
pelo câncer de pele \cite{skin_cancer_screening}. A detecção precoce é essencial para a garantia da efetividade do
tratamento, pois a taxa de sobrevida dos pacientes tende a cair ao longo do avanço da doença
\cite{skin_cancer_survival}.

A dificuldade da identificação precoce de lesões de pele tem um componente socioeconômico. Em um estudo feito por
\textcite{skin_cancer_socioeconomic}, observa-se que indivíduos com um grau menor de escolaridade têm um prognóstico
pior, pois apresentam tumores em estágios mais avançados. Esta correlação pode indicar que a desigualdade resulta no
atraso do diagnóstico e consequentemente na redução da taxa de sucesso do tratamento.
\textcite{santos2020desigualdades} também constata que a saúde dos brasileiros está associada à renda e classe
econômica.

No Brasil, uma parte da atenção primária é feita por \ac{ACS}, que possuem um treinamento básico para atender a
população e dar orientações simples de saúde. Esses são os profissionais que normalmente estão em contato com as
regiões mais carentes. A identificação da causa de uma lesão de pele está além das responsabilidades de um \ac{ACS}
\cite{filgueiras2011agente}. Isso pode ser problemático, pois para realizar a triagem de um caso de lesão de pele, se
faz necessária a análise de um dermatologista, que nem sempre está disponível em regiões carentes.

Considerando este cenário, seria útil a existência de um serviço com uma boa usabilidade que analisasse imagens de
pacientes e recomendasse uma forma de triagem, mitigando a carência de dermatologistas. As capacidades de classificar e
descrever imagens precisamente tornam \acp{MLLM} bons candidatos para a fundação desse serviço
\cite{mllm_success_rate}.

Neste trabalho, o \ac{LLaMA}-3.2 foi escolhido como o modelo base para o desenvolvimento, pois ele tem o código aberto
e também é relevante no estado da arte em tarefas visuais e textuais \cite{dubey2024llama}. O modelo foi treinado com
técnicas de \textit{fine-tuning} sobre o conjunto de dados do \ac{STT/SC} para a classificação e lesões de pele e a
geração de laudos sobre imagens.

Com o objetivo utilizar os recursos computacionais disponíveis eficientemente, o \textit{fine-tuning} foi feito com
técnicas baseadas em \ac{PEFT}. Este método permite a redução do número de parâmetros usados no processo, resultando em
uma utilização menor de recursos \cite{peft}. Especificamente, utilizaram-se os métodos \ac{QLoRA} e \ac{LoRA} para o
treinamento do modelo.

\section{Objetivos}

O objetivo principal deste trabalho é treinar um \ac{LLM} multimodal com métodos de \textit{fine-tuning} para a
classificação de lesões de pele e geração de laudos, avaliando o desempenho dos modelos resultantes e comparando as
diferentes técnicas de treinamento.

\subsection*{Objetivos Específicos}

\begin{itemize}
    \item Treinar o modelo \ac{LLaMA}-3.2 com as técnicas de \textit{fine-tuning} \ac{QLoRA} e \ac{LoRA};
    \item Comparar os modelos treinados com o \ac{LLaMA}-3.2;
    \item Avaliar os modelos treinados e comparar as diferentes técnicas de \textit{fine-tuning} entre si.
\end{itemize}

\section{Organização do Trabalho}

% TODO: Rever isso no final

O trabalho apresenta inicialmente os conceitos teóricos sobre lesões de pele e \acp{MLLM}, abordando questões sobre a
arquitetura e treinamento destes modelos. Logo em seguida, é feita uma apresentação e discussão sobre trabalhos
correlatos. Depois, são apresentados os experimentos, abordando os dados e tecnologias utilizadas, características dos
treinamentos e os resultados. Por fim, é feita a discussão sobre os resultados e possíveis trabalhos futuros são
apresentados.
