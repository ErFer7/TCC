\chapter{Introdução}

Lesões de pele são áreas anormais ou danificadas da pele, causadas por diversos fatores, como condições ambientais, infecções ou doenças graves, incluindo o câncer de
pele. A identificação precisa da causa é fundamental para o tratamento adequado, porém, devido à semelhança entre os sintomas, diagnosticar corretamente as doenças que
afetam a pele pode ser uma tarefa complexa. Além disso, o prognóstico pode se agravar significativamente se o tratamento não for iniciado cedo o suficiente
\cite{habif2015clinical}.

A importância do cuidado com lesões de pele é evidenciada pelo impacto do câncer de pele. No Brasil, 30\% dos tumores malignos registrados são atribuídos a essa doença
\cite{skin_cancer_in_brazil}. Os tipos mais comuns, o carcinoma basocelular e o carcinoma epidermoide, apresentam baixa letalidade, mas podem deixar sequelas
significativas devido ao tratamento. Já o melanoma, embora menos frequente, é altamente letal e corresponde a cerca de 75\% das mortes causadas pelo câncer de pele
\cite{skin_cancer_screening}. A detecção precoce é essencial para a garantia da efetividade do tratamento, pois a taxa de sobrevida dos pacientes tende a cair ao longo
do avanço da doença \cite{skin_cancer_survival}.

A dificuldade da identificação precoce de lesões de pele tem um componente socioeconômico. Em um estudo feito por \textcite{skin_cancer_socioeconomic}, observa-se que
indivíduos com um grau menor de escolaridade têm um prognóstico pior, pois apresentam tumores em estágios mais avançados. Esta correlação pode indicar que a desigualdade
resulta no atraso do diagnóstico e consequentemente na redução da taxa de sucesso do tratamento. \textcite{santos2020desigualdades} também constata que a saúde dos
brasileiros está associada à renda e classe econômica.

No Brasil, uma parte da atenção primária é feita por \ac{ACS}, que possuem um treinamento básico para atender a população e dar orientações simples de saúde. Esses
são os profissionais que normalmente estão em contato com as regiões mais carentes. A identificação da causa de uma lesão de pele está além das responsabilidades de um
\ac{ACS} \cite{filgueiras2011agente}. Isso pode ser problemático, pois para realizar a triagem de um caso de lesão de pele, se faz necessária a análise de um dermatologista, que nem sempre está
disponível em regiões carentes.

Considerando este cenário, seria útil a existência de um serviço com uma boa usabilidade que analisasse imagens de pacientes e recomendasse uma forma de triagem,
mitigando a carência de dermatologistas. As capacidades de classificar lesões de pele precisamente e boa usabilidade com interação conversacional tornam \acp{MLLM}
bons candidatos para a fundação desse serviço \cite{mllm_success_rate}.

Neste trabalho, o \ac{LLaMA} 3.2 foi escolhido como o modelo base para o desenvolvimento, pois ele tem o código aberto e também é relevante no estado da arte
\cite{dubey2024llama}.

% TODO: Alterar essa parte com o contexto do fine-tuning completo.

Com este modelo, é possível realizar um \textit{fine-tuning}, adaptando-o para a classificação de lesões de pele. Com o objetivo utilizar os recursos computacionais
disponíveis eficientemente, o \textit{fine-tuning} será feito com técnicas baseadas em \ac{PEFT}. Este método permite a redução do número de parâmetros usados no
processo, resultando em uma utilização menor de recursos \cite{peft}. Especificamente, planeja-se utilizar métodos como \ac{LoRA} e \ac{QLoRA} para a adaptação do modelo.

Com as adaptações feitas no \ac{LLaMA} 3.2, espera-se obter um sistema confiável de classificação de lesões de pele.

% TODO: Avaliar se isso está certo para o TCC 1

\section{Objetivos}

O objetivo deste trabalho é desenvolver um sistema de classificação de lesões de pele utilizando um \ac{LLM} multimodal. Serão utilizados métodos de \textit{fine-tuning}
para adaptar o modelo base. Além disso, planeja-se analisar a eficiência entre diferentes técnicas de adaptação.

\subsection*{Objetivos Específicos}

\begin{itemize}
    \item Avaliar a eficiência de diferentes métodos de \textit{fine-tuning} e a precisão dos modelos resultantes;
    \item Obter um modelo adaptado que possua uma precisão satisfatória na classificação de lesões de pele;
    \item Comparar o modelo adaptado com outros \acp{MLLM}.
\end{itemize}
