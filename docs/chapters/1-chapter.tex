\chapter{Introdução}

Lesões de pele são indicativos de diversas doenças, como infecções de baixo risco ou até mesmo câncer. Cerca de 30\% dos tumores malignos registrados no Brasil são
causados pelo câncer de pele \cite{skin_cancer_in_brazil}. Os tipos mais comuns desta doença são o carcinoma basocelular e o carcinoma epidermoide, que mesmo tendo uma
baixa letalidade, podem causar sequelas expressivas por conta do tratamento. O melanoma é o menos frequente, mas possui uma letalidade muito maior, causando 75\% das mortes
por câncer de pele \cite{skin_cancer_screening}.

A detecção precoce do câncer de pele é essencial para a garantia da efetividade do tratamento, pois a taxa de sobrevida dos pacientes tende a cair ao longo do avanço
da doença \cite{skin_cancer_survival}. Observa-se que indivíduos com um grau menor de escolaridade têm um prognóstico pior, pois apresentam tumores em estágios mais
avançados \cite{skin_cancer_socioeconomic}. Esta correlação pode indicar que a falta de conhecimento sobre a doença ou problemas socioeconômicos podem resultar no atraso
do diagnóstico e consequentemente na redução da taxa de sucesso do tratamento.

Considerando este cenário, o desenvolvimento de um sistema automatizado de fácil utilização de classificação de lesões de pele pode ser útil na identificação precoce de
melanomas e outros tipos de lesões de pele. \ac{MLLMs} conseguem obter uma precisão significativa na classificação de imagens de diagnósticos médicos, o que torna viável
o uso destes modelos para o desenvolvimento deste trabalho \cite{mllm_success_rate}.

O \ac{LLaVa} foi escolhido como o modelo base devido à limitação de recursos para o treinamento de um novo modelo e por ter o código disponível publicamente. Este
modelo combina o codificador visual \ac{CLIP ViT-L/14} e o \ac{LLM} \textit{Vicuna} \cite{llava}.

Com este modelo, é possível realizar um \textit{fine-tuning}, adaptando-o para a classificação de lesões de pele. Com o objetivo utilizar os recursos computacionais
disponíveis eficientemente, o \textit{fine-tuning} será feito com técnicas baseadas em \ac{PEFT}. Este método permite a redução do número de parâmetros usados no processo,
resultando em uma utilização menor de recursos \cite{peft}. Especificamente, planeja-se utilizar métodos como \ac{LoRa} ou outros similares para a adaptação do modelo.

Com as adaptações feitas no \ac{LLaVa}, espera-se obter um sistema confiável de classificação de lesões de pele.

\section{Objetivos}

O objetivo deste trabalho é desenvolver um sistema de classificação de lesões de pele utilizando um \ac{LLM} multimodal. Serão utilizados métodos de \textit{fine-tuning}
para adaptar o modelo base. Além disso, planeja-se analisar a eficiência entre diferentes técnicas de adaptação.

\subsection*{Objetivos Específicos}

\begin{itemize}
    \item Avaliar a eficiência de diferentes métodos de \textit{fine-tuning} e a precisão dos modelos resultantes;
    \item Obter um modelo adaptado baseado no \ac{LLaVa} que possua uma precisão satisfatória na classificação de lesões de pele;
    \item Comparar o modelo adaptado com outros \ac{MLLMs}.
\end{itemize}
